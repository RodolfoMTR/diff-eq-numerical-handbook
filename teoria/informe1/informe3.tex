Es un programa que calcula las cotas del error del método de Euler para una ecuación diferencial ordinaria (EDO) dada. El método de Euler es un método numérico para resolver EDOs, y las cotas del error proporcionan una estimación de la precisión de la solución aproximada en comparación con la solución exacta.
\begin{theo}{}{}
Suponga que \textit{f} es continua y satisface la condición de Lipschitz con constante $L$ en 
\begin{equation*}
D=\{(t,y)|a\leq t\leq b \text{ y }-\infty<y<\infty\}
\end{equation*}
y que existe una constante $M$ con
\begin{equation*}
|y''(t)|\leq M,\text{ para todas las }t\in[a,b],
\end{equation*}
donde $y(t)$ denota la única solución para el problema de valor inicial
\begin{equation*}
y'=f(t,y),\quad a\leq t\leq b,\quad y(a)=\alpha.
\end{equation*}
Sean $w_0,w_1,...,w_N$ las aproximaciones generadas por el método de Euler para un entero positivo $N$. Entonces, para cada $i=0,1,2,\cdots,N,$
\begin{equation}
|y(t_i)-w_i|\leq\frac{hM}{2L}\left[e^{L(t_i-a)}-1\right]
\end{equation}
\end{theo}
\textbf{Ejemplo:} Encontrar las cotas del error de Euler para la ecuación diferencial
\begin{equation*}
y' = t - y, \quad y(0) = 0 \text{ con } 0 \leq t \leq 1
\end{equation*}
\textbf{Solución:}

La solución de la ecuación diferencial es
\begin{equation*}
y = t - 1 + e^{-t}
\end{equation*}
Creamos un script llamado \textbf{cotas\_euler.m} que llama a la función \textbf{euler.m} del informe anterior.

\textbf{Código en Matlab/Octave:}
\begin{matlabcode}
%a,b: intervalo donde se define la EDO
%M : constante
%N : constante
%L : constante de Lipschitz
%f : f(x,y) de y'=f(x,y)
%y : solución analitica de y'=f(x,y) y0=alpha
a=0; b=1; %t en [a,b]
f = @(t,y) t-y; %función f(t,y) de la EDO
M=1;
L=1;
N=3; %para h=1/3
y = @(t) t-1+exp(-t); %sol. analítica de la EDO
y0=0; %condición inicial
%-----------------------------
% Cotas del error de Euler
%-----------------------------
h=(b-a)/N;
[t,w]=euler(a,b,N,y0,f);
fprintf('%8s %15s %20s\n', 't_i', '|y_i - w_i|', 'cota de error');
for i = 1:N
    ti = t(i);
    err = abs(y(ti) - w(i));
    bound = (h * M / (2 * L)) * abs(exp(L * (ti - a)) - 1);
    fprintf('%8.4f %15.6f %20.6f\n', ti, err, bound);
end
\end{matlabcode}
\textbf{Manual:}
\begin{itemize}
    \item Abrir el archivo \textbf{cotas\_euler.m} en Matlab/Octave.
    \item En las lineas 
    \begin{verbatim}
a=0; b=1; %t en [a,b]
f = @(t,y) t-y; %función f(t,y) de la EDO
M=1;
L=1;
N=3; %para h=1/3
y = @(t) t-1+exp(-t); %sol. analítica de la EDO
y0=0; %condición inicial
    \end{verbatim}
    Se ingresa el intervalo de la EDO, la función $f(t,y)$, las constantes $M$ y $L$, y la solución analítica de la EDO.
    \item En la línea \textbf{N=3;} se define el número de pasos a realizar.
    \item Ejecutar el script con el comando \textbf{cotas\_euler2} en la consola de Matlab/Octave.
\end{itemize}
\textbf{Corrida del Programa:}
\begin{matlaboutput}
>> cotas_euler
     t_i     |y_i - w_i|        cota de error
  0.0000        0.000000             0.000000
  0.3333        0.049865             0.065935
  0.6667        0.068973             0.157956
\end{matlaboutput}
