El método Crank-Nicolson es un esquema numérico implícito para resolver ecuaciones en derivadas parciales (EDP) de tipo parabolico, como la ecuación de difusión o calor. Es una combinación de los métodos explícito e implícito, usando el promedio entre ambos en cada paso temporal. Para $\theta=1/2$, se obtiene el método clásico de Crank-Nicolson, que es estable y de segundo orden en el tiempo y espacio.

\textbf{Variables:}

\begin{itemize}
    \item $a$, $b$: Extremos del intervalo espacial.
    \item $n$: Número de subintervalos espaciales.
    \item $h$: Tamaño de paso espacial, $h=(b-a)/n$.
    \item $tt$: Tiempo final de la simulación.
    \item $k$: Tamaño de paso temporal.
    \item $\alpha$: Parámetro de difusión ($\alpha^2$).
    \item $L$: Número de Fourier, $L=\alpha k/h^2$.
    \item $x$: Vector de nodos espaciales.
    \item $w0$: Condición inicial en los nodos.
    \item $A$, $B$: Matrices del sistema lineal para el método.
    \item $BB$: Matriz que almacena la solución en cada paso temporal.
    \item $m$: Número de pasos temporales.
\end{itemize}

\textbf{Programa en Matlab/Octave:}
\begin{matlabcode}
%Crank_nicolson general para theta=1/2
a=0;b=1;n=10;h=(b-a)/n;
tt=1;k=1/12; alp=1; %alp: alpha cuadrado
L=alp*k/h^2;BB=A=B=[];bb=[];
x=a:h:b;w0=x-x.^2; BB(:,1)=w0;
w0=w0(2:n)';
A(1,1)=1+L;A(1,2)=-L/2;
B(1,1)=1-L;B(1,2)=L/2;
for i=2:n-2
    A(i,i-1)=-L/2;A(i,i)=1+L;A(i,i+1)=-L/2;
    B(i,i-1)=L/2;B(i,i)=1-L;B(i,i+1)=L/2;
end
i=n-1;
A(i,i-1)=-L/2;A(i,i)=1+L;
B(i,i-1)=L/2;B(i,i)=1-L;
m=ceil(tt/k);
for i=1:m
    wx=B*w0;wx=A\wx;BB(:,i+1)=[0 wx' 0]';
    w0=wx;
end
BB
mesh(BB)
\end{matlabcode}

\textbf{Manual:}

Para ejecutar el programa, siga estos pasos:

1. Copie el código Matlab/Octave en un archivo llamado `crank\_nicolson.m`.
2. Abra Matlab o GNU Octave y navegue hasta el directorio donde guardó el archivo.
3. Ejecute el archivo escribiendo `crank\_nicolson` en la consola.
4. El resultado será la matriz `BB`, que contiene la solución numérica en cada paso temporal, y una gráfica de la evolución de la solución.

La matriz `BB` muestra cómo la condición inicial evoluciona en el tiempo bajo el método Crank-Nicolson. La gráfica generada (`mesh(BB)`) permite visualizar la difusión de la solución en el dominio espacio-tiempo.
\textbf{Corrida del Programa:}
\begin{matlaboutput}
BB =

 Columns 1 through 8:

        0        0        0        0        0        0        0        0
   0.0900   0.0272   0.0181   0.0033   0.0043  -0.0002   0.0014  -0.0005
   0.1600   0.0594   0.0290   0.0102   0.0052   0.0019   0.0008   0.0005
   0.2100   0.0875   0.0368   0.0158   0.0062   0.0030   0.0009   0.0007
   0.2400   0.1061   0.0419   0.0191   0.0073   0.0034   0.0013   0.0006
   0.2500   0.1126   0.0437   0.0201   0.0077   0.0035   0.0015   0.0005
   0.2400   0.1061   0.0419   0.0191   0.0073   0.0034   0.0013   0.0006
   0.2100   0.0875   0.0368   0.0158   0.0062   0.0030   0.0009   0.0007
   0.1600   0.0594   0.0290   0.0102   0.0052   0.0019   0.0008   0.0005
   0.0900   0.0272   0.0181   0.0033   0.0043  -0.0002   0.0014  -0.0005
        0        0        0        0        0        0        0        0

 Columns 9 through 13:

        0        0        0        0        0
   0.0006  -0.0004   0.0003  -0.0003   0.0002
   0.0000   0.0002  -0.0001   0.0001  -0.0001
   0.0000   0.0002  -0.0001   0.0001  -0.0000
   0.0002   0.0001   0.0001  -0.0000   0.0000
   0.0003   0.0000   0.0001  -0.0000   0.0000
   0.0002   0.0001   0.0001  -0.0000   0.0000
   0.0000   0.0002  -0.0001   0.0001  -0.0000
   0.0000   0.0002  -0.0001   0.0001  -0.0001
   0.0006  -0.0004   0.0003  -0.0003   0.0002
        0        0        0        0        0
\end{matlaboutput}
\begin{figure}[ht]
    \centering
    \includegraphics[width=0.8\textwidth]{im12.pdf}
    \caption{Solución de la EDP con Crank-Nicolson}
    \label{fig:solucion_edp12}
\end{figure}
