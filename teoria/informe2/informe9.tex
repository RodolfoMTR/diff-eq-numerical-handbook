Este programa resuelve un problema de valor en la frontera no lineal mediante el método de diferencias finitas iterativo.  
La ecuación diferencial a resolver es:
\[
y'' - \frac{y^2}{10} = t,\quad \text{con } y(0) = 0,\quad y(1) = 1
\]
Este problema no es lineal debido al término \( y^2 \), por lo que se emplea una linealización iterativa, donde se sustituye \( y^2 \) por una función \( s(t)y \), y se actualiza \( s(t) \) con el valor de \( y \) obtenido en cada iteración. De esta forma, en cada paso se resuelve un sistema lineal que aproxima la solución del problema no lineal.

---

\textbf{Variables:}

\begin{itemize}
    \item \texttt{p}: Coeficiente del término \( y' \), que es cero en este caso.
    \item \texttt{a}, \texttt{b}: Extremos del intervalo en el que se resuelve la EDO.
    \item \texttt{n}: Número de subintervalos para la discretización. Esto define \( n+1 \) nodos.
    \item \texttt{h}: Tamaño del paso, calculado como \( h = (b - a)/n \).
    \item \texttt{ya}, \texttt{yb}: Condiciones de frontera \( y(a) = ya \) y \( y(b) = yb \).
    \item \texttt{st}: Vector que almacena los valores de la función \( s(t) \), que se actualiza con los valores de \( y \) en cada iteración.
    \item \texttt{u}: Término auxiliar igual a \( h^2 \), utilizado para simplificar expresiones.
    \item \texttt{A}: Matriz del sistema lineal generado por el método de diferencias finitas.
    \item \texttt{bb}: Vector del segundo miembro del sistema, dependiendo de la función fuente \( r(x) = t \) y de las condiciones de frontera.
    \item \texttt{cc}: Solución del sistema lineal en cada iteración; representa la aproximación de la solución \( y \) en los nodos interiores.
    \item \texttt{t}: Vector de nodos del intervalo \([a, b]\).
\end{itemize}

\textbf{Programa en Matlab/Octave:}
\begin{matlabcode}
%No lineal Iteraciones Metodo de diferencias finitas
%y''-y^2/10=t  y(0)=0; y(1)=1
%y''-s(t)y/10=t; donde s(t)=y
p=0;n=4;%q=st
a=0;b=1;h=(b-a)/n;A=[];st=a:h:b;st=st(2:n);
ya=0;yb=1;u=h.^2;
A(1,1)=st(1)*u-2;A(1,2)=1+0*h/2;bb(1)=(a+h)*u-ya;
for j=1:5
for i=2:(n-2)
        A(i,i-1)=1;A(i,i)=st(i)*u-2;A(i,i+1)=1;
        bb(i)=u*(i*h);
end%i
A(n-1,n-2)=1;A(n-1,n-1)=st(n-1)*u-2;
bb(n-1)=(a+(n-1)*h)*u-yb;
[A,bb'] 
cc=A\bb';
if norm(cc-st)<0.01 break;end
st=cc/10;
end;%j
cc=[ya cc' yb]
t=a:h:b;
plot(t,cc);
\end{matlabcode}
\textbf{Manual:}

1. Guarde el programa en un archivo llamado \texttt{nolineal.m}.

2. Ejecute el script desde la consola de Octave o Matlab con:
\begin{verbatim}
>> nolineal
\end{verbatim}

3. El código implementa un proceso iterativo para resolver una EDO no lineal utilizando el método de diferencias finitas.

4. En cada iteración se linealiza la ecuación con una estimación de \( s(t) = y \), se forma el sistema lineal correspondiente, y se resuelve. El proceso continúa hasta que el cambio entre iteraciones es menor a 0.01.

5. Al final, se grafica la solución aproximada en el intervalo \([0,1]\).


\textbf{Corrida del Programa:}
\begin{matlaboutput}
octave:10> nolineal
ans =

  -1.9844   1.0000        0   0.0156
   1.0000  -1.9688   1.0000   0.0312
        0   1.0000  -1.9531  -0.9531

ans =

  -1.9844   1.0000        0   0.0156
   1.0000  -1.9971   1.0000   0.0312
        0   1.0000  -1.9954  -0.9531

ans =

  -1.9844   1.0000        0   0.0156
   1.0000  -1.9972   1.0000   0.0312
        0   1.0000  -1.9956  -0.9531

ans =

  -1.9844   1.0000        0   0.0156
   1.0000  -1.9972   1.0000   0.0312
        0   1.0000  -1.9956  -0.9531

ans =

  -1.9844   1.0000        0   0.0156
   1.0000  -1.9972   1.0000   0.0312
        0   1.0000  -1.9956  -0.9531

cc =

        0   0.2148   0.4419   0.6991   1.0000

\end{matlaboutput}
\begin{figure}[ht]
    \centering
    \includegraphics[width=0.5\textwidth]{im4_2.pdf}
\end{figure}
