El objetivo de esta sección es mostrar cómo graficar la solución de una ecuación diferencial ordinaria (EDO) utilizando Matlab/Octave. Para ello, se resolverá una EDO específica y se graficará su solución junto con algunos puntos evaluados en la misma. Sea la EDO con condición inicial
\begin{equation}
y'+2y=x\quad ,\, y(0)=0\quad x\in[0,1]
\end{equation}
La solución de la ecuación diferencial es
\begin{equation}
y=\frac{2x-1}{4}+\frac{e^{-2x}}{4}
\end{equation}
Queremos encontrar los valores en $y(1),\,y(1/2)\,,y(\pi/4)$, de ahí, graficar junto a la solución $y$ en \textbf{Matlab/octave}.Creamos el siguiente script y lo guadadamos con el nombre que guste:
\begin{matlabcode}
y = @(x) (2*x-1)/4+exp(-2*x)/4; %definimos como funcion solucion de la EDO
% evaluamos la funcion en los puntos deseados
y(1)
y(1/2)
y(pi/4)
%GRAFICA
x=0:0.01:1; %dominio de la solucion y el paso h=0.01
plot(x,y(x),1,y(1),'ro',1/2,y(1/2),'ro',pi/4,y(pi/4),'ro')
\end{matlabcode}
Ejecutamos el script, y nos debe dar el siguiente resultado
\begin{matlaboutput}
    ans = 0.2838
    ans = 0.091970
    ans = 0.1947    
\end{matlaboutput}
con la gráfica en el intervalo $[0,1]$ con los puntos calculados
\begin{figure}[ht]
\centering
\includegraphics[width=7cm]{figure_0.pdf}
\caption{Gráfica de la solución}
\end{figure}
