Este informe presenta la resolución de una ecuación diferencial ordinaria (EDO) de segundo orden utilizando el método de colocación de base. Se emplean funciones base y sus derivadas para aproximar la solución, implementando el procedimiento en Matlab/Octave.

\textbf{Variables:} \\
\begin{itemize}
    \item \texttt{u1, u2}: Funciones base utilizadas para aproximar la solución.
    \item \texttt{du1, du2}: Derivadas de las funciones base.
    \item \texttt{ddu1, ddu2}: Segundas derivadas de las funciones base.
    \item \texttt{p, q}: Coeficientes de la EDO.
    \item \texttt{A}: Matriz de coeficientes del sistema lineal.
    \item \texttt{b}: Vector de términos independientes.
    \item \texttt{x}: Puntos de colocación y vector de evaluación.
    \item \texttt{c}: Coeficientes de la combinación lineal de las bases.
    \item \texttt{y}: Aproximación de la solución de la EDO.
\end{itemize}

\textbf{Programa en Matlab/Octave:}
\begin{matlabcode}
%19/06/2025
%colocacion base
%bases
u1=@(x) (1-x).*x;
u2=@(x) u1(x).*x;
%derivadas
du1=@(x) (1-2*x);
du2=@(x) 2*x-3*x.*x;
%segundas derivadas
ddu1=@(x) -2;
ddu2=@(x) 2-6*x;
%y''+y'-2y=x y(0)=0=y(1)
p=1;q=-2;
A=[];b=[];x=[1/2 4/3];
A(1,1)=ddu1(x(1))+p*du1(x(1))+q*u1(x(1));
A(1,2)=ddu2(x(1))+p*du2(x(1))+q*u2(x(1));
A(2,1)=ddu1(x(2))+p*du1(x(2))+q*u1(x(2));
A(2,2)=ddu2(x(2))+p*du2(x(2))+q*u2(x(2));
b(1)=x(1);b(2)=x(2);
c=inv(A)*b';
x=0:0.01:1;
y=c(1)*u1(x)+c(2)*u2(x);
plot(x,y)
hold on
%tarea, hacer que funciones para {1/4 1/2 3/4}
%u3=(1-x)x^3
\end{matlabcode}

\textbf{Manual:}

Para ejecutar el programa en Matlab o Octave, siga estos pasos:

\begin{enumerate}
    \item Copie el código proporcionado en una nueva ventana de script.
    \item Guarde el archivo con extensión \texttt{.m}, por ejemplo, \texttt{colocacion\_base.m}.
    \item Ejecute el script en el entorno Matlab/Octave. El programa calculará la aproximación de la solución de la EDO y mostrará la gráfica correspondiente.
    \item Puede modificar los puntos de colocación en el vector \texttt{x} y agregar nuevas funciones base para experimentar con diferentes aproximaciones.
\end{enumerate}

El gráfico generado muestra la solución aproximada obtenida mediante el método de colocación de base.
\textbf{Corrida del Programa:}
\begin{figure}[ht]
    \centering
    \includegraphics[width=0.8\textwidth]{im1.pdf}
    \caption{Solución de la EDO}
    \label{fig:solucion_edo}
\end{figure}
