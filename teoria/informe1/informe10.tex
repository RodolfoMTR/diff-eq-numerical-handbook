Es un ejemplo de uso de la función \texttt{runge11.m} que se definió en el informe 8. Esta función implementa el método de Runge-Kutta de orden 4 para resolver ecuaciones diferenciales ordinarias (EDO). En este caso, se utiliza para resolver la EDO:
\begin{equation*}
    y'=t-ty \quad y(0)=\frac{1}{2}\quad t\in[0,1]
\end{equation*}
que se definio en el informe 9 mediante la función \texttt{ff.m}.

Se crea un nuevo programa llamado \textbf{llamadorRK.m} que llama a la función \texttt{runge11} y tiene la siguiente forma:
\begin{matlabcode}
    a=0;b=1;n=4;h=(b-a)/n;
    y0=1/2;
    y35=runge11(a,b,h,4,y0)
\end{matlabcode}
Se ejecuta el programa \textbf{llamadorRK.m} y se obtiene el siguiente resultado:
\begin{matlaboutput}
    >> llamadorRK

    y35 =
    
        0.5000    0.5154    0.5588    0.6226    0.6967
\end{matlaboutput}
