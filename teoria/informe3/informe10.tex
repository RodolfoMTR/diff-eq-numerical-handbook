Este programa resuelve una ecuación diferencial parcial parabólica utilizando el método de diferencias finitas en Matlab/Octave. Calcula la evolución de la solución en una malla discreta, mostrando los resultados numéricos y gráficos.

\textbf{Variables:}

\begin{itemize}
   \item \texttt{ff}: función fuente dependiente de \(x\).
   \item \texttt{a}, \texttt{b}: extremos del intervalo espacial.
   \item \texttt{n}: número de subintervalos espaciales.
   \item \texttt{h}: tamaño de paso espacial.
   \item \texttt{k}: tamaño de paso temporal.
   \item \texttt{al}: parámetro de difusión.
   \item \texttt{lam}: coeficiente de estabilidad (\(\lambda = \alpha k / h^2\)).
   \item \texttt{A}: matriz del sistema lineal.
   \item \texttt{bb}: vector del lado derecho.
   \item \texttt{liz}, \texttt{lde}: condiciones de frontera izquierda y derecha.
   \item \texttt{B}: matriz para almacenar soluciones en cada paso temporal.
   \item \texttt{x}, \texttt{x1}: vectores de nodos espaciales.
   \item \texttt{c}: solución en cada paso temporal.
\end{itemize}

\textbf{Programa en Matlab/Octave:}
\begin{matlabcode}
%parabolicas 2025/07/03
%este programa funciona con casi todo
ff=@(x) 2*x-x.^2;
a=0;b=2;n=4;h=(b-a)/n;
k=1/10;al=4;lam=al*k/(h.^2);
A=[];bb=[];B=A;liz=0;lde=1;
A(1,1)=1+2*lam;A(1,2)=-lam;
for i=2:n-2
    A(i-1,i)=-lam;A(i,i)=1+2*lam;A(i,i+1)=-lam;
end
i=n-1;A(i,i-1)=-lam;A(i,i)=1+2*lam
x=a:h:b;x1=x(2:n);bb=ff(x1)';bb(n-1)=bb(n-1)+lam*lde;bb(1)=bb(1)+lam*liz;
B(:,1)=[0 bb' 0];
for i=1:10
    c=A\bb
    B(:,i+1)=[0 c' 0];
    bb=c;
end
mesh(B)
\end{matlabcode}

\textbf{Manual:}

\begin{itemize}
   \item Ingrese la función fuente \texttt{ff} según el problema.
   \item Defina los extremos del intervalo espacial (\texttt{a}, \texttt{b}) y el número de subintervalos (\texttt{n}).
   \item Calcule el tamaño de paso espacial (\texttt{h}) y temporal (\texttt{k}), así como el parámetro de difusión (\texttt{al}).
   \item Construya la matriz del sistema lineal (\texttt{A}) y el vector del lado derecho (\texttt{bb}), considerando las condiciones de frontera (\texttt{liz}, \texttt{lde}).
   \item Inicialice la matriz de soluciones (\texttt{B}) y los vectores de nodos espaciales (\texttt{x}, \texttt{x1}).
   \item Resuelva iterativamente el sistema lineal para cada paso temporal, actualizando la solución y almacenando los resultados en \texttt{B}.
   \item Visualice la evolución de la solución utilizando la función \texttt{mesh}.
\end{itemize}
\textbf{Corrida del Programa:}
\begin{matlaboutput}
A =

   4.2000  -1.6000        0
        0   4.2000  -1.6000
        0  -1.6000   4.2000

c =

   0.3797
   0.5279
   0.7606

c =

   0.1771
   0.2277
   0.2678

c =

   0.077164
   0.091841
   0.098760

c =

   0.032109
   0.036058
   0.037250

c =

   0.012976
   0.013995
   0.014201

c =

   5.1484e-03
   5.4045e-03
   5.4399e-03

c =

   2.0191e-03
   2.0824e-03
   2.0885e-03

c =

   7.8610e-04
   8.0157e-04
   8.0263e-04

c =

   3.0466e-04
   3.0841e-04
   3.0859e-04

c =

   1.1773e-04
   1.1864e-04
   1.1867e-04
\end{matlaboutput}
\begin{figure}[ht]
    \centering
    \includegraphics[width=0.8\textwidth]{im10.pdf}
    \caption{Solución de la EDP}
    \label{fig:solucion_edp10}
\end{figure}
