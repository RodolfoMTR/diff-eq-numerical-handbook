Es un ejemplo de uso del método de Moulton para resolver la ecuación diferencial ordinaria (EDO):
\begin{equation*}
    y'=t-ty \quad y(0)=\frac{1}{2}\quad t\in[0,1]
\end{equation*}
Este método es un método implícito de predicción-corrección que utiliza el valor de la función en el paso siguiente para corregir el valor actual. Por ese motivo, se utiliza la función \texttt{runge11.m} que se definió en el informe 8.

Sea crea un nuevo programa con el nombre de \textbf{moulton.m}.

\textbf{Código en Matlab/Octave:}
\begin{matlabcode}
    clear all;
    y0=1/2; %condición inicial
    a=0;b=1;% intervalo
    n=4; % número de pasos
    %----------------------
    t=a;
    h=(b-a)/n;
    yt=[];yt(1)=y0;
    u=runge11(a,b,h,1,y0);
    yt=u;
    t=a+h;
    y0=yt(2);
    for i=2:n
      v=y0;
      for k=1:5
        v1=y0+h/12*(5*ff(t+h,v)+8*ff(t,yt(i))-ff(t-h,yt(i-1)));
        if abs(v1-v)<0.0001 
            break
        end
        v=v1;
      end%k
      y0=v1;t=t+h;yt(i+1)=y0;
    end%i
    u=a:h:b;
    disp([u',yt']);
\end{matlabcode}
\textbf{Manual:}
\begin{itemize}
    \item Abrir el archivo \textbf{moulton.m} en Matlab/Octave.
    \item En el programa \texttt{ff.m} se define la EDO a resolver.
    \item En las lineas 
    \begin{verbatim}
    y0=1/2; %condición inicial
    a=0;b=1;% intervalo
    n=4; % número de paso
    \end{verbatim}
    se definen los parámetros de la EDO a resolver
    \item En la línea \textbf{n=4;} se define el número de pasos a realizar.
    \item Ejecutar el script con el comando \textbf{moulton} en la consola de Matlab/Octave.
\end{itemize}
\textbf{Corrida del Programa:}
\begin{matlaboutput}
>> Moulton
    0    0.5000
0.2500    0.5154
0.5000    0.5586
0.7500    0.6223
1.0000    0.6965
\end{matlaboutput}
