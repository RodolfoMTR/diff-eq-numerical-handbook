\documentclass{article}
\usepackage{amsmath,amsfonts,amssymb,amsthm}
\usepackage{mathrsfs}
\usepackage[spanish]{babel}
\newcommand{\R}{\mathbb R}
\newcommand{\A}{\mathcal A}
\newcommand{\B}{\mathcal B}
\newcommand{\Rc}{\mathcal R}
\newcommand{\N}{\mathbb N}
\newcommand{\T}{\mathscr{T}}
\usepackage[utf8]{inputenc}
\setlength{\parindent}{0cm}%sin identación
\usepackage{graphicx}
\usepackage{float}
\usepackage[left=2.54cm, right=2.54cm,top=2.54cm,bottom=2.54cm,a4paper,heightrounded]{geometry}
\usepackage{tcolorbox}%for theorems
\tcbuselibrary{theorems}%for theorems
\usepackage[T1]{fontenc}
\usepackage{lmodern}
\usepackage{color}
\usepackage[table]{xcolor}
\usepackage{hyperref}
\usepackage{matlab}
\sloppy
%for theorems
\newtcbtheorem[number within=section]{theo}{Teorema}%
{colback=green!5,colframe=green!35!black,fonttitle=\bfseries}{th}
\newtcbtheorem[number within=section]{prop}{Proposición}%
{colback=green!5,colframe=green!35!black,fonttitle=\bfseries}{prop}

\newtcbtheorem[number within=section]{cor}{Corolario}%
{colback=magenta!5,colframe=magenta!35!black,fonttitle=\bfseries}{cor}

\newtcbtheorem[number within=section]{lema}{Lema}%
{colback=yellow!5,colframe=yellow!35!black,fonttitle=\bfseries}{lem}

\newtcbtheorem[number within=section]{deff}{Definición}%
{colback=blue!5,colframe=blue!70!black,fonttitle=\bfseries}{deff}

\newtheorem*{remark}{Observación}
\newtheorem{exercise}{Ejercicio}
\newenvironment{prueba}{\paragraph{Demostración:}\quad\\}{\hfill$\blacksquare$}
% Define todas las rutas aquí
\graphicspath{ {./informe1/media1/} {./informe2/media2/} {./informe3/media3/} }
%------titulo
\title{Manual de métodos numéricos para EDO's y EDP's}
\author{Rodolfo M. Turpo R.}
\date{08 de agosto de 2025}
%-----
\begin{document}
\maketitle
\tableofcontents
\newpage

\section{Gráfica de Solución de una EDO}
Este informe presenta la resolución de una ecuación diferencial ordinaria (EDO) de segundo orden utilizando el método de colocación de base. Se emplean funciones base y sus derivadas para aproximar la solución, implementando el procedimiento en Matlab/Octave.

\textbf{Variables:} \\
\begin{itemize}
    \item \texttt{u1, u2}: Funciones base utilizadas para aproximar la solución.
    \item \texttt{du1, du2}: Derivadas de las funciones base.
    \item \texttt{ddu1, ddu2}: Segundas derivadas de las funciones base.
    \item \texttt{p, q}: Coeficientes de la EDO.
    \item \texttt{A}: Matriz de coeficientes del sistema lineal.
    \item \texttt{b}: Vector de términos independientes.
    \item \texttt{x}: Puntos de colocación y vector de evaluación.
    \item \texttt{c}: Coeficientes de la combinación lineal de las bases.
    \item \texttt{y}: Aproximación de la solución de la EDO.
\end{itemize}

\textbf{Programa en Matlab/Octave:}
\begin{matlabcode}
%19/06/2025
%colocacion base
%bases
u1=@(x) (1-x).*x;
u2=@(x) u1(x).*x;
%derivadas
du1=@(x) (1-2*x);
du2=@(x) 2*x-3*x.*x;
%segundas derivadas
ddu1=@(x) -2;
ddu2=@(x) 2-6*x;
%y''+y'-2y=x y(0)=0=y(1)
p=1;q=-2;
A=[];b=[];x=[1/2 4/3];
A(1,1)=ddu1(x(1))+p*du1(x(1))+q*u1(x(1));
A(1,2)=ddu2(x(1))+p*du2(x(1))+q*u2(x(1));
A(2,1)=ddu1(x(2))+p*du1(x(2))+q*u1(x(2));
A(2,2)=ddu2(x(2))+p*du2(x(2))+q*u2(x(2));
b(1)=x(1);b(2)=x(2);
c=inv(A)*b';
x=0:0.01:1;
y=c(1)*u1(x)+c(2)*u2(x);
plot(x,y)
hold on
%tarea, hacer que funciones para {1/4 1/2 3/4}
%u3=(1-x)x^3
\end{matlabcode}

\textbf{Manual:}

Para ejecutar el programa en Matlab o Octave, siga estos pasos:

\begin{enumerate}
    \item Copie el código proporcionado en una nueva ventana de script.
    \item Guarde el archivo con extensión \texttt{.m}, por ejemplo, \texttt{colocacion\_base.m}.
    \item Ejecute el script en el entorno Matlab/Octave. El programa calculará la aproximación de la solución de la EDO y mostrará la gráfica correspondiente.
    \item Puede modificar los puntos de colocación en el vector \texttt{x} y agregar nuevas funciones base para experimentar con diferentes aproximaciones.
\end{enumerate}

El gráfico generado muestra la solución aproximada obtenida mediante el método de colocación de base.
\textbf{Corrida del Programa:}
\begin{figure}[ht]
    \centering
    \includegraphics[width=0.8\textwidth]{im1.pdf}
    \caption{Solución de la EDO}
    \label{fig:solucion_edo}
\end{figure}

\section{Método de Euler}
Este informe presenta la resolución de una ecuación diferencial ordinaria (EDO) utilizando el método de colocación con funciones base polinomiales. Se desarrolla el procedimiento teórico y se implementa el algoritmo en Matlab/Octave para obtener la solución aproximada.

\textbf{Variables:}

\begin{itemize}
    \item $u_1(x), u_2(x)$: Funciones base utilizadas en el método de colocación.
    \item $du_1(x), du_2(x)$: Derivadas de las funciones base.
    \item $ddu_1(x), ddu_2(x)$: Segundas derivadas de las funciones base.
    \item $A$: Matriz de coeficientes del sistema lineal generado por el método.
    \item $b$: Vector de términos independientes.
    \item $p, q$: Coeficientes de la EDO.
    \item $c$: Vector de coeficientes de la combinación lineal de las funciones base.
    \item $x$: Vector de puntos en el intervalo de solución.
    \item $y$: Solución aproximada de la EDO en los puntos $x$.
\end{itemize}

\textbf{Programa en Matlab/Octave:}
\begin{matlabcode}
%19/06/2025
u1=@(x) (1-x).*x;u2=@(x) u1(x).*x;
du1=@(x) (1-2.*x);du2=@(x) 2.*x-3.*x.*x;
ddu1=@(x) -2;ddu2=@(x) 2-6.*x;
A=[]; b=[]; p=1; q=-2;
pp=@(x) (ddu1(x)+p.*du1(x)+q.*u1(x)).*u1(x);
A(1,1)=quad(pp,0,1);
pp=@(x) (ddu2(x)+p.*du2(x)+q.*u2(x)).*u1(x);
A(1,2)=quad(pp,0,1);
pp=@(x) (ddu1(x)+p.*du1(x)+q.*u1(x)).*u2(x);
A(2,1)=quad(pp,0,1);
pp=@(x) (ddu2(x)+p.*du2(x)+q.*u2(x)).*u2(x);
A(2,2)=quad(pp,0,1);
pp=@(x) x.*u1(x);
b(1)=quad(pp,0,1);
pp=@(x) x.*u2(x);
b(2)=quad(pp,0,1);
c=A\b';
x=0:0.001:1;
y=c(1).*u1(x)+c(2).*u2(x);
plot(x,y,'m');
grid;
\end{matlabcode}

\textbf{Manual:}

\begin{enumerate}
    \item Definir las funciones base $u_1(x)$ y $u_2(x)$, junto con sus derivadas primeras y segundas.
    \item Plantear la ecuación diferencial ordinaria y los coeficientes $p$ y $q$.
    \item Construir la matriz de coeficientes $A$ y el vector de términos independientes $b$ mediante la integración de los productos adecuados.
    \item Resolver el sistema lineal $A c = b$ para obtener los coeficientes $c$ de la solución aproximada.
    \item Evaluar la solución aproximada $y(x)$ en el intervalo deseado utilizando la combinación lineal de las funciones base.
    \item Graficar la solución obtenida para visualizar el comportamiento de la EDO resuelta.
\end{enumerate}
\textbf{Corrida del Programa:}

\begin{figure}[ht]
    \centering
    \includegraphics[width=0.8\textwidth]{im2.pdf}
    \caption{Solución de la EDO}
    \label{fig:solucion_edo2}
\end{figure}

\section{Cotas del error de Euler}
Es un \textit{script} que calcula las cotas del error del método de Euler para una ecuación diferencial ordinaria (EDO) dada. El método de Euler es un método numérico para resolver EDOs, y las cotas del error proporcionan una estimación de la precisión de la solución aproximada en comparación con la solución exacta.
\begin{theo}{}{}
Suponga que \textit{f} es continua y satisface la condición de Lipschitz con constante $L$ en 
\begin{equation*}
D=\{(t,y)|a\leq t\leq b \text{ y }-\infty<y<\infty\}
\end{equation*}
y que existe una constante $M$ con
\begin{equation*}
|y''(t)|\leq M,\text{ para todas las }t\in[a,b],
\end{equation*}
donde $y(t)$ denota la única solución para el problema de valor inicial
\begin{equation*}
y'=f(t,y),\quad a\leq t\leq b,\quad y(a)=\alpha.
\end{equation*}
Sean $w_0,w_1,...,w_N$ las aproximaciones generadas por el método de Euler para un entero positivo $N$. Entonces, para cada $i=0,1,2,\cdots,N,$
\begin{equation}
|y(t_i)-w_i|\leq\frac{hM}{2L}\left[e^{L(t_i-a)}-1\right]
\end{equation}
\end{theo}
\textbf{Ejemplo:} Encontrar las cotas del error de Euler para la ecuación diferencial
\begin{equation*}
y' = t - y, \quad y(0) = 0 \text{ con } 0 \leq t \leq 1
\end{equation*}
\textbf{Solución:}

La solución analítica de la ecuación diferencial es
\begin{equation*}
y = t - 1 + e^{-t}
\end{equation*}
y se quiere comparar con la solución aproximada.

Creamos un script llamado \textbf{cotas\_euler.m} que llama a la función \textbf{euler.m} de la sección anterior.

\begin{matlabcode}
%a,b: intervalo donde se define la EDO
%M : constante
%N : numero de pasos para h=(b-a)/N
%L : constante de Lipschitz
%f : f(x,y) de y'=f(x,y)
%y : solucion analitica del problema de valor inicial
%y0=alpha : condicion inicial
a=0; b=1; %t en [a,b]
f = @(t,y) t-y; %funcion f(t,y) de la EDO
M=1;
L=1;
N=3; %para h=1/3
y = @(t) t-1+exp(-t); %sol. analitica de la EDO
y0=0; %condicion inicial
%-----------------------------
% Cotas del error de Euler
%-----------------------------
h=(b-a)/N;
[t,w]=euler(a,b,N,y0,f);
fprintf('%8s %15s %20s\n', 't_i', '|y_i - w_i|', 'cota de error');
for i = 1:N
    ti = t(i);
    err = abs(y(ti) - w(i));
    bound = (h * M / (2 * L)) * abs(exp(L * (ti - a)) - 1);
    fprintf('%8.4f %15.6f %20.6f\n', ti, err, bound);
end
\end{matlabcode}
\textbf{Resultados:}
\begin{matlaboutput}
>> cotas_euler
     t_i     |y_i - w_i|        cota de error
  0.0000        0.000000             0.000000
  0.3333        0.049865             0.065935
  0.6667        0.068973             0.157956
\end{matlaboutput}

\section{Método de taylor}
Este programa implementa el método del disparo para resolver un problema de valor en la frontera (boundary value problem, BVP) para un sistema de ecuaciones diferenciales de primer orden. El método convierte el BVP en un problema de valor inicial (IVP), que luego se resuelve usando el método de Runge-Kutta de orden 4.

En particular, el sistema resuelto es:
\[
\begin{cases}
x' = 2x + 3y \\
y' = \frac{2}{3}x + 3y \\
x(0) = 0, \quad y(0) = 1, \quad t \in [0,1]
\end{cases}
\]

Este tipo de técnica es común cuando se conocen condiciones en los extremos del intervalo y se requiere una solución aproximada de alta precisión.

---

\textbf{Variables:}

\begin{itemize}
    \item \texttt{a}, \texttt{b}: Extremos del intervalo de integración en el tiempo (\( t \in [a, b] \)).
    \item \texttt{x}, \texttt{y}: Condiciones iniciales \( x(a) \), \( y(a) \).
    \item \texttt{n}: Número de subintervalos en los que se divide el intervalo \([a, b]\); define la precisión del método.
    \item \texttt{h}: Tamaño del paso \( h = \frac{b - a}{n} \).
    \item \texttt{t}: Variable del tiempo que avanza en cada iteración.
    \item \texttt{x0}, \texttt{y0}: Valores actuales de las variables \( x \) e \( y \) en cada paso.
    \item \texttt{xt}, \texttt{yt}: Vectores donde se almacenan los valores aproximados de \( x(t) \) y \( y(t) \).
    \item \texttt{kij}: Coeficientes intermedios del método de Runge-Kutta para calcular la solución numérica, tanto para \( x \) como para \( y \).
    \item \texttt{fff(t,x,y)}: Función que representa la derivada de \( x \) en el sistema (es decir, \( x' = f(x,y) \)).
    \item \texttt{fffg(t,x,y)}: Función que representa la derivada de \( y \) en el sistema (es decir, \( y' = g(x,y) \)).
    \item \texttt{yy}: Matriz de dos filas: la primera contiene los valores de \( x(t) \), y la segunda los de \( y(t) \).
\end{itemize}

---


\textbf{Programa en Matlab/Octave:}
\begin{matlabcode}
% x' = 2x   + 3y
% y' = 2x/3 + 3y ;       x(0)=0;  y(0)=1;  t en [0,1]
function yy=disparo_r_k_11(a,b,x,y,n)
h=(b-a)/n; t=a;
x0=x; y0=y; xt(1)=x0; yt(1)=y0;
% k11 k12 k13 k14 para fff(t)
% k21  k22  k23 k24 para fffg(t)
for i=1:n
  k11 = h*fff(t, x0,y0);                  k21=h*fffg(t,x0,y0);
  k12 = h*fff(t+h/2, x0+k11/2,y0+k21/2);  k22=h*fffg(t+h/2,x0+k11/2,y0+k21/2);
  k13 = h*fff(t+h/2, x0+k12/2,y0+k22/2);  k23=h*fffg(t+h/2,x0+k12/2,y0+k22/2);
  k14 = h*fff(t+h, x0+k13,y0+k23);        k24=h*fffg(t+h,x0+k13,y0+k23);
  xs=x0+(k11 + 2*(k12+k13) + k14)/6; xt(i+1)=xs;
  ys=y0+(k21 + 2*(k22+k23) + k24)/6; yt(i+1)=ys;
  t=t+h; x0=xs; y0=ys;
end;
yy=[xt ; yt];

\end{matlabcode}

\textbf{Manual:}
Se escribe un nuevo script de nombre disparo\_r\_k\_11.m, que es el programa que resuelve el problema de valor de frontera con el método de Runge-Kutta de orden 4. Este programa usa las funciones fff(t,x,y) y fffg(t,x,y) que se escribieron en los informes siguientes.

\textbf{Corrida del Programa:}
Este programa no se ejecuta directamente, sino que se llama desde otro programa de nombre llamador.m.

\section{Runge Kutta}
\subsection{Runge kutta de orden 2}
Este programa implementa el método de diferencias finitas para resolver una ecuación diferencial parcial (EDP) de tipo parabólico, utilizando Matlab/Octave. Se calcula la evolución temporal de la solución en una malla discreta, mostrando los resultados numéricos y la gráfica correspondiente.

\textbf{Variables:} \\
\begin{itemize}
  \item \texttt{la}: Parámetro de estabilidad, calculado como $(4 \times 1/65) / (0.5^2)$.
  \item \texttt{A}: Matriz de coeficientes del sistema lineal generado por el método de diferencias finitas.
  \item \texttt{n}: Número de divisiones espaciales (nodos menos uno).
  \item \texttt{h}: Tamaño de paso espacial, calculado como $(2-0)/n$.
  \item \texttt{x}: Vector de posiciones espaciales.
  \item \texttt{w0}: Vector de condiciones iniciales en los nodos internos.
  \item \texttt{B}: Matriz que almacena la evolución de la solución en cada paso temporal.
  \item \texttt{w1}: Vector de solución en el siguiente paso temporal.
\end{itemize}

\textbf{Programa en Matlab/Octave:}
\begin{matlabcode}
la=(4*1/65)/(0.5^2);A=[];n=4;h=(2-0)/n;
A(1,1)=1-2*la;A(1,2)=la;
for i=2:n-2
    A(i,i-1)=la;A(i,i)=1-2*la;A(i,i+1)=la;
end
A(n-1,n-2)=la;A(n-1,n-1)=1-2*la;
x=0:h:2;
w0=x.^2-2*x;w0=w0';B=[];w0=w0(2:n);B(:,1)=w0;
for i=1:10
    w1=A*w0
    B(:,i+1)=w1;
    w0=w1;
end
mesh(B)
\end{matlabcode}

\textbf{Manual:}

Para ejecutar el programa, siga estos pasos:

\begin{enumerate}
  \item Abra Matlab o Octave en su computadora.
  \item Copie el código proporcionado en una nueva ventana de script y guárdelo como \texttt{dif\_finitas.m}.
  \item Ejecute el script presionando el botón de ejecución o escribiendo \texttt{dif\_finitas} en la consola.
  \item Observe la salida numérica en la consola, que muestra la evolución de la solución en cada paso temporal.
  \item Se generará una gráfica de tipo \texttt{mesh} que representa la evolución de la solución en la malla discreta.
\end{enumerate}

Asegúrese de tener instalado Matlab o GNU Octave y de que el archivo se encuentre en el directorio de trabajo.
\textbf{Corrida del Programa:}
\begin{matlaboutput}
w1 =

  -0.6269
  -0.8769
  -0.6269

w1 =

  -0.5341
  -0.7538
  -0.5341

w1 =

  -0.4567
  -0.6457
  -0.4567

w1 =

  -0.3908
  -0.5527
  -0.3908

w1 =

  -0.3345
  -0.4730
  -0.3345

w1 =

  -0.2862
  -0.4048
  -0.2862

w1 =

  -0.2450
  -0.3464
  -0.2450

w1 =

  -0.2096
  -0.2965
  -0.2096

w1 =

  -0.1794
  -0.2537
  -0.1794

w1 =

  -0.1535
  -0.2171
  -0.1535
\end{matlaboutput}
\begin{figure}[ht]
    \centering
    \includegraphics[width=0.8\textwidth]{im5.pdf}
    \caption{Solución de la EDP}
    \label{fig:solucion_edp5}
\end{figure}

\subsection{Runge kutta de orden 3}
Es un método numérico para resolver ecuaciones diferenciales ordinarias (EDO) de la forma
\begin{equation*}
    y' = f(t,y), \quad y(t_0) = y_0\quad  a\leq t\leq b
\end{equation*}
donde $f(t,y)$ es una función continua y $y_0$ es el valor inicial de la solución en el punto $t_0$. El método Runge-Kutta de orden 3 es una extensión del método de Runge-Kutta de orden 2, proporcionando una mayor precisión en la aproximación de la solución.

Para $n=3$, es posible efectuar un desarrollo similar al del método de segundo orden. Una versión comun que se obtiene es
\begin{equation}
    y_{i+1}=y_i+\frac{1}{6}(k_1+4k_2+k_3)h
\end{equation}
donde
\begin{align*}
    k_1&=f(x_i,y_i)\\
    k_2&=f(x_i+\frac{1}{2}h,y_i+\frac{1}{2}k_1h)\\
    k_3&=f(x_i+h,y_i-k_1h+2k_2h)
\end{align*}
Sea la EDO que no se puede resolver analiticamente
\begin{equation*}
    y'=t-ty^{1.5}
\end{equation*}
Entonces utilizando Runge-Kutta de orden 3, creamos un archivo en matlab/octave llamado RKorden3.m, y escribimos el siguiente codigo.

\textbf{Código en Matlab/Octave:}
\begin{matlabcode}
    %Runge Kutta de orden 3
    y1 = @(t,y) t-y^(3/2); %Ec. Diff. Ordinaria a solucionar
    y0=1; %condición inicial
    a=1;b=2; %Intervalos
    N=8;
    h = (b-a)/N;
    yt=zeros(1,N);yt(1)=y0;
    a1=a;
    for i=1:N
        k1=h*y1(a1,y0);
        k2=h*y1(a1+h/2,y0+k1/2);
        k3=h*y1(a1+h,y0+k1+2*k2);
        yp=y0+(k1+4*k2+k3)/6;yt(i+1)=yp;
        fprintf('%d| yn=%f| k1=%f| k2=%f| k3=%f y_n+1=%f\n',i,y0,k1,k2,k3,yp)
        y0=yp; a1=a1+h;
    end
      t=a:h:b;
      pa=polyfit(t,yt,4); % minimos cuadrados
      pt=polyval(pa,t);
      plot(t,pt,t,yt,'.r','MarkerSize',10)
\end{matlabcode}
\textbf{Manual:}
\begin{itemize}
    \item Abrir el archivo \textbf{Rkorden3.m} en Matlab/Octave.
    \item En las lineas 
    \begin{verbatim}
y1 = @(t,y) t-y^(3/2); %Ec. Diff. Ordinaria a solucionar
y0=1; %condición inicial
a=1;b=2; %Intervalos
    \end{verbatim}
    Se ingresa la ecuación diferencial ordinaria a solucionar, la condición inicial y el intervalo de solución.
    \item En la línea \textbf{N=8;} se define el número de pasos a realizar.
    \item Ejecutar el script con el comando \textbf{Rkorden3} en la consola de Matlab/Octave.
\end{itemize}
\textbf{Corrida del Programa:}
\begin{matlaboutput}
    >> RKorden3
    1| yn=1.000000| k1=0.000000| k2=0.007812| k3=0.012684 y_n+1=1.007322
    2| yn=1.007322| k1=0.014250| k2=0.020719| k3=0.019251 y_n+1=1.026718
    3| yn=1.026718| k1=0.026207| k2=0.031522| k3=0.024512 y_n+1=1.056186
    4| yn=1.056186| k1=0.036193| k2=0.040504| k3=0.028619 y_n+1=1.093991
    5| yn=1.093991| k1=0.044469| k2=0.047899| k3=0.031722 y_n+1=1.138622
    6| yn=1.138622| k1=0.051252| k2=0.053909| k3=0.033965 y_n+1=1.188764
    7| yn=1.188764| k1=0.056736| k2=0.058714| k3=0.035482 y_n+1=1.243277
    8| yn=1.243277| k1=0.061090| k2=0.062477| k3=0.036398 y_n+1=1.301176
\end{matlaboutput}
\begin{figure}[ht]
    \centering
    \includegraphics[width = 10cm]{figure2_info5.pdf}
\end{figure}

\subsection{Runge kutta de orden 4}
Este programa resuelve un problema de valor en la frontera (BVP) para una ecuación diferencial lineal de segundo orden utilizando el método de diferencias finitas. En lugar de construir la matriz del sistema de forma automática, en este caso se ha calculado manualmente la matriz de coeficientes \( A \) y el vector del lado derecho \( b \), lo que simplifica la implementación y permite concentrarse en el concepto del método.

Se supone que la EDO tiene la forma general:
\[
y'' + p(x)y' + q(x)y = r(x), \quad y(0) = 0,\ y(1) = 1
\]
y que el dominio \([0,1]\) se ha discretizado con \( n = 4 \) puntos (3 nodos interiores).

---

\textbf{Variables:}

\begin{itemize}
    \item \texttt{A}: Matriz de coeficientes del sistema lineal generado por el método de diferencias finitas, con los valores ya ingresados manualmente.
    \item \texttt{b}: Vector del lado derecho del sistema, calculado según las condiciones de frontera y los valores de la función \( r(x) \) en los nodos internos.
    \item \texttt{y}: Vector solución del sistema lineal \( A \cdot y = b \), que contiene las aproximaciones de la función en los nodos interiores. Se completa con las condiciones de frontera: \( y(0) = 0 \) y \( y(1) = 1 \).
    \item \texttt{x}: Vector que contiene los puntos del dominio equiespaciados en el intervalo \([0,1]\), incluyendo los extremos.
\end{itemize}

\textbf{Programa en Matlab/Octave:}
\begin{matlabcode}
A=[-15/8 11/8 0; 5/8 -15/8 11/8;0 5/8 -15/8];
b=[1/64 1/32 -85/64];
y=inv(A)*b'
y=[0 y' 1];
x=0:1/4:1;
plot(x,y)
\end{matlabcode}

\textbf{Manual:}

1. Guarde el código en un archivo llamado \texttt{dif\_finitas\_manual.m}.

2. Ejecute desde la consola de Octave o Matlab con:
\begin{verbatim}
>> dif_finitas_manual
\end{verbatim}

3. El programa resuelve el sistema lineal ingresado manualmente y grafica la función aproximada \( y(x) \) en el intervalo \([0,1]\), incluyendo los valores de frontera.

4. La matriz \( A \) y el vector \( b \) se han deducido manualmente a partir de la discretización por diferencias finitas y de las condiciones de contorno.

\textbf{Corrida del Programa:}
\begin{matlaboutput}
y =

   0.7091
   0.9783
   1.0344

\end{matlaboutput}
\begin{figure}[ht]
\centering
\includegraphics[width=0.5\textwidth]{im2_2.pdf}
\end{figure}

\subsection{Runge kutta como función en matlab/octave}
Este programa resuelve un problema de valor en la frontera (BVP) para una ecuación diferencial de segundo orden utilizando el método de diferencias finitas, donde la matriz de coeficientes y el vector del segundo miembro se generan automáticamente dentro del código.

El problema considerado es de la forma:
\[
y'' + p y' + q y = r(x), \quad \text{en } [a,b], \quad y(a) = y_a, \ y(b) = y_b
\]
En este caso particular:

- \( p = 3 \), \( q = 2 \)

- \( r(x) = x \) (función lineal)

- Condiciones de frontera: \( y(0) = 0 \), \( y(1) = 1 \)

Se discretiza el intervalo \([0,1]\) en \(n = 10\) subintervalos y se construye un sistema lineal \( A \cdot y = b \), que se resuelve para encontrar las aproximaciones de \( y(x) \) en los nodos interiores.

---

\textbf{Variables:}

\begin{itemize}
    \item \texttt{p}, \texttt{q}: Coeficientes de la EDO en los términos \( y' \) y \( y \) respectivamente.
    \item \texttt{r(x)}: Función fuente del lado derecho de la ecuación. En este caso, es \( r(x) = x \).
    \item \texttt{a}, \texttt{b}: Extremos del intervalo donde se resuelve el problema.
    \item \texttt{n}: Número de subintervalos de la discretización (hay \(n-1\) nodos interiores).
    \item \texttt{ya}, \texttt{yb}: Valores de las condiciones de frontera en \(x = a\) y \(x = b\), respectivamente.
    \item \texttt{h}: Paso de la malla, \( h = \frac{b-a}{n} \).
    \item \texttt{A}: Matriz de coeficientes del sistema lineal construida con base en la discretización de la EDO.
    \item \texttt{b}: Vector del segundo miembro construido en función de \( r(x) \) y de las condiciones de frontera.
    \item \texttt{cc}: Solución del sistema lineal, concatenada con los valores de frontera para obtener la solución completa.
    \item \texttt{t}: Vector que contiene los nodos de la discretización.
\end{itemize}


\textbf{Programa en Matlab/Octave:}
\begin{matlabcode}
% y''+py''+qy=r(x)
p=3;q=2;a=0;b=1;n=10;
ya=0;yb=1;
%------------------
h=(b-a)/n;
A=[];b1=b;
A(1,1)=2*h.^2-2;
A(1,2)=1+3*h/2;
b(1)=(a+h)*h.^2-(1-3*h/2)*ya;
for i=2:(n-2)
        A(i,i-1)=1-3*h/2;
        A(i,i)=2*h*h-2;
        A(i,i+1)=1+3*h/2;
        b(i)=h.^2*(i*h);
end%i
A(n-1,n-2)=1-3*h/2;A(n-1,n-1)=2*h.^2-2;b(n-1)=(a+(n-1)*h)*h.^2-(1+3*h/2)*yb;
[A b']
cc=A\b';cc=[ya cc' yb];
t=a:h:b1;
plot(t,cc,'LineWidth', 2);  % Cambia el 2 por el grosor que desees
\end{matlabcode}

\textbf{Manual:}

1. Guarde el código en un archivo llamado \texttt{dif\_finitas\_auto.m}.

2. Ejecute en Octave o Matlab con:
\begin{verbatim}
>> dif_finitas_auto
\end{verbatim}

3. El programa construye automáticamente la matriz \( A \) y el vector \( b \) a partir de los coeficientes \( p \), \( q \), y de la función \( r(x) = x \).

4. Resuelve el sistema \( A \cdot y = b \) e incorpora las condiciones de frontera para graficar la solución aproximada.

\textbf{Corrida del Programa:}
\begin{matlaboutput}
ans =

  -1.9800   1.1500        0        0        0        0        0        0        0   0.0010
   0.8500  -1.9800   1.1500        0        0        0        0        0        0   0.0020
        0   0.8500  -1.9800   1.1500        0        0        0        0        0   0.0030
        0        0   0.8500  -1.9800   1.1500        0        0        0        0   0.0040
        0        0        0   0.8500  -1.9800   1.1500        0        0        0   0.0050
        0        0        0        0   0.8500  -1.9800   1.1500        0        0   0.0060
        0        0        0        0        0   0.8500  -1.9800   1.1500        0   0.0070
        0        0        0        0        0        0   0.8500  -1.9800   1.1500   0.0080
        0        0        0        0        0        0        0   0.8500  -1.9800  -1.1410
\end{matlaboutput}
\begin{figure}[ht]
    \centering
    \includegraphics[width=0.5\textwidth]{im3_2.pdf}
\end{figure}

\subsection{función \textbf{ff.m} para Runge-Kutta}
Este informe presenta la resolución numérica de una ecuación diferencial parcial parabólica utilizando el método de diferencias finitas implícito. El objetivo es aproximar la solución de la EDP en un dominio definido, mostrando el procedimiento, el código implementado y los resultados obtenidos.

\textbf{Variables:}

\begin{itemize}
   \item \texttt{ff}: función fuente del problema, definida como \( f(x) = 2x - x^2 \).
   \item \texttt{a}, \texttt{b}: extremos del intervalo espacial (\( a = 0 \), \( b = 2 \)).
   \item \texttt{n}: número de subintervalos espaciales (\( n = 10 \)).
   \item \texttt{h}: tamaño de paso espacial (\( h = (b-a)/n \)).
   \item \texttt{k}: tamaño de paso temporal (\( k = 1/100 \)).
   \item \texttt{al}: parámetro de difusión (\( \alpha = 4 \)).
   \item \texttt{lam}: parámetro auxiliar (\( \lambda = \alpha k / h^2 \)).
   \item \texttt{A}: matriz de coeficientes del sistema lineal.
   \item \texttt{bb}: vector del lado derecho en cada paso temporal.
   \item \texttt{B}: matriz que almacena las soluciones en cada paso temporal.
   \item \texttt{x}, \texttt{x1}: vectores de nodos espaciales.
   \item \texttt{c}: solución del sistema en cada iteración temporal.
\end{itemize}

\textbf{Programa en Matlab/Octave:}
\begin{matlabcode}
%parabolicas 2025/07/03
%este programa funciona con casi todo
ff=@(x) 2*x-x.^2;
a=0;b=2;n=10;h=(b-a)/n;
k=1/100;al=4;lam=al*k/(h.^2);
A=[];bb=[];B=A;
A(1,1)=1+2*lam;A(1,2)=-lam;
for i=2:n-2
    A(i-1,i)=-lam;A(i,i)=1+2*lam;A(i,i+1)=-lam;
end
i=n-1;A(i,i-1)=-lam;A(i,i)=1+2*lam
x=a:h:b;x1=x(2:n);bb=ff(x1)';B(:,1)=[0 bb' 0];
for i=1:10
    c=A\bb
    B(:,i+1)=[0 c' 0];
    bb=c;
end
mesh(B)
\end{matlabcode}

\textbf{Manual:}

Para resolver la ecuación diferencial parcial parabólica por el método de diferencias finitas implícito, se sigue el siguiente procedimiento:

1. **Discretización del dominio:** Se divide el intervalo espacial \([a, b]\) en \(n\) subintervalos de tamaño \(h\), y el tiempo en pasos de tamaño \(k\).

2. **Formulación de la ecuación discreta:** La ecuación parabólica se aproxima usando diferencias finitas centradas en el espacio y hacia adelante en el tiempo, resultando en un sistema lineal para cada paso temporal.

3. **Construcción de la matriz de coeficientes:** Se arma la matriz tridiagonal \(A\) que representa la relación entre los nodos interiores en cada paso temporal, incorporando el parámetro de difusión \(\alpha\) y el parámetro auxiliar \(\lambda\).

4. **Condiciones iniciales y de frontera:** Se establecen los valores iniciales y las condiciones de frontera en los extremos del dominio.

5. **Resolución iterativa:** En cada paso temporal, se resuelve el sistema lineal \(A c = bb\) para obtener la solución en los nodos interiores, actualizando el vector de solución para el siguiente paso.

6. **Almacenamiento y visualización:** Las soluciones en cada paso temporal se almacenan en la matriz \(B\) y se visualizan mediante una gráfica de malla (mesh).

Este método es estable y adecuado para problemas parabólicos, permitiendo aproximar la evolución temporal de la solución en el dominio definido.
\textbf{Corrida del Programa:}
\begin{matlaboutput}
A =

 Columns 1 through 8:

   3.0000  -1.0000        0        0        0        0        0        0
        0   3.0000  -1.0000        0        0        0        0        0
        0        0   3.0000  -1.0000        0        0        0        0
        0        0        0   3.0000  -1.0000        0        0        0
        0        0        0        0   3.0000  -1.0000        0        0
        0        0        0        0        0   3.0000  -1.0000        0
        0        0        0        0        0        0   3.0000  -1.0000
        0        0        0        0        0        0        0   3.0000
        0        0        0        0        0        0        0  -1.0000

 Column 9:

        0
        0
        0
        0
        0
        0
        0
  -1.0000
   3.0000

c =

   0.2400
   0.3601
   0.4402
   0.4806
   0.4817
   0.4450
   0.3750
   0.2850
   0.2150

c =

   0.1451
   0.1952
   0.2256
   0.2365
   0.2288
   0.2049
   0.1696
   0.1337
   0.1162

c =

   0.082651
   0.102877
   0.113426
   0.114729
   0.107720
   0.094317
   0.078090
   0.064687
   0.060312

c =

   0.045221
   0.053013
   0.056163
   0.055062
   0.050456
   0.043649
   0.036629
   0.031797
   0.030703

c =

   0.024015
   0.026825
   0.027462
   0.026224
   0.023609
   0.020371
   0.017464
   0.015762
   0.015488

c =

   1.2463e-02
   1.3374e-02
   1.3298e-02
   1.2431e-02
   1.1071e-02
   9.6025e-03
   8.4368e-03
   7.8467e-03
   7.7783e-03

c =

   6.3508e-03
   6.5892e-03
   6.3933e-03
   5.8820e-03
   5.2146e-03
   4.5732e-03
   4.1172e-03
   3.9148e-03
   3.8977e-03

c =

   3.1890e-03
   3.2161e-03
   3.0592e-03
   2.7844e-03
   2.4712e-03
   2.1991e-03
   2.0241e-03
   1.9553e-03
   1.9510e-03

c =

   1.5826e-03
   1.5588e-03
   1.4601e-03
   1.3212e-03
   1.1792e-03
   1.0665e-03
   1.0004e-03
   9.7710e-04
   9.7603e-04

c =

   7.7811e-04
   7.5175e-04
   6.9649e-04
   6.2931e-04
   5.6673e-04
   5.2093e-04
   4.9628e-04
   4.8841e-04
   4.8815e-04
\end{matlaboutput}
\begin{figure}[ht]
    \centering
    \includegraphics[width=0.8\textwidth]{im9.pdf}
    \caption{Solución de la EDP}
    \label{fig:solucion_edp9}
\end{figure}

\subsection{Ejemplo de uso de la función \textbf{runge11}}
Es un ejemplo de uso de la función \texttt{runge11.m} que se definió en el informe 8. Esta función implementa el método de Runge-Kutta de orden 4 para resolver ecuaciones diferenciales ordinarias (EDO). En este caso, se utiliza para resolver la EDO:
\begin{equation*}
    y'=t-ty \quad y(0)=\frac{1}{2}\quad t\in[0,1]
\end{equation*}
que se definio en el informe 9 mediante la función \texttt{ff.m}.

Se crea un nuevo programa llamado \textbf{llamadorRK.m} que llama a la función \texttt{runge11} y tiene la siguiente forma:
\begin{matlabcode}
    a=0;b=1;n=4;h=(b-a)/n;
    y0=1/2;
    y35=runge11(a,b,h,4,y0)
\end{matlabcode}
Se ejecuta el programa \textbf{llamadorRK.m} y se obtiene el siguiente resultado:
\begin{matlaboutput}
    >> llamadorRK

    y35 =
    
        0.5000    0.5154    0.5588    0.6226    0.6967
\end{matlaboutput}

\subsection{Runge kutta para sistemas de EDO's}
Es un program que resuelve un sistema de EDO's de orden 1:
\begin{equation*}
    \begin{cases}
        x' = 2x + 3y \\
        y' = \frac{2}{3}x + 3y
    \end{cases}
\quad x(0)=0, y(0)=1 \quad t\in[0,1]
\end{equation*}
Se crea un nuevo programa con el nombre de \textbf{RK\_sistemas.m}.

\textbf{Código en Matlab/Octave:}
\begin{matlabcode}
    %R-K de orden 4 para sistemas
    clear all
    %format long
    format shortG
    %---------------------
    x1 = @(t,x,y) 2*x+3*y;
    y1 = @(t,x,y) 2/3*x+3*y;
    a=0;b=1;n=10;
    t=a;x0=0;y0=1;
    %---------------------
    h=(b-a)/n; %paso
    xt=[];yt=[];
    xt(1)=x0;yt(1)=y0;
    for i=1:n
        k11=h*x1(t,x0,y0);
        k12=h*y1(t,x0,y0);
        k21=h*x1(t+h/2,x0+k11/2,y0+k12/2);
        k22=h*y1(t+h/2,x0+k11/2,y0+k12/2);
        k31=h*x1(t+h/2,x0+k21/2,y0+k22/2);
        k32=h*y1(t+h/2,x0+k21/2,y0+k22/2);
        k41=h*x1(t+h,x0+k31,y0+k32);
        k42=h*y1(t+h,x0+k31,y0+k32);
        xs=x0+(k11+2*(k21+k31)+k41)/6;
        ys=y0+(k12+2*(k22+k32)+k42)/6;
        xt(i+1)=xs;yt(i+1)=ys;
        t=t+h;x0=xs;y0=ys;
    end%i
    u=0:h:1;
    disp([u' xt' yt'])    
\end{matlabcode}
\textbf{Manual:}
\begin{itemize}
    \item Abrir el archivo \textbf{RK\_sistemas.m} en Matlab/Octave.
    \item En las lineas 
    \begin{verbatim}
    x1 = @(t,x,y) 2*x+3*y;
    y1 = @(t,x,y) 2/3*x+3*y;
    a=0;b=1;n=10;
    t=a;x0=0;y0=1;
    \end{verbatim}
    Se define el sistema de EDO's a resolver y sus parámetros.
    \item Ejecutar el script con el comando \textbf{RK\_sistemas} en la consola de Matlab/Octave.
\end{itemize}
\textbf{Corrida del Programa:}
\begin{matlaboutput}
    >> RK_sistemas
    0            0            1
  0.1      0.38656       1.3629
  0.2       1.0039       1.8906
  0.3       1.9696        2.663
  0.4         3.46       3.7985
  0.5       5.7381       5.4741
  0.6        9.197       7.9535
  0.7       14.424        11.63
  0.8       22.295       17.089
  0.9       34.118       25.205
    1       51.846       37.283
\end{matlaboutput}

\section{Método de Adams-Bashforth}
\subsection{Adams-Bashforth de orden 2}
Este conjunto de programas resuelve un problema de valor en la frontera no lineal (BVP) de la forma:

\[
y'' + 3y' + 2y = t, \quad y(a)=ya, \quad y(b)=yb,
\]

utilizando el método del disparo, que transforma el problema BVP en un problema de valor inicial (IVP).  
La idea es proponer una pendiente inicial \( y'(a) = s \), resolver el sistema con esta condición inicial mediante el método de Runge-Kutta, y ajustar \( s \) iterativamente hasta que la solución cumpla con la condición de frontera en \( b \).

Se implementa una versión no lineal del método del disparo, en la cual se aplica el método de la secante para aproximar la pendiente correcta que satisface la condición final. La no linealidad del problema requiere un ajuste iterativo de esta pendiente hasta obtener el valor deseado en \( y(b) \).

---

\textbf{Variables:}

\begin{itemize}
    \item \texttt{a}, \texttt{b}: Extremos del intervalo del dominio de la solución.
    \item \texttt{n}: Número de subintervalos para la discretización del intervalo.
    \item \texttt{h}: Tamaño del paso, calculado como \( h = (b-a)/n \).
    \item \texttt{y1}, \texttt{y2}: Pendientes iniciales propuestas para el disparo, que se actualizan con el método de la secante.
    \item \texttt{k}: Parámetro que representa el control de la no homogeneidad del sistema (presencia del término \( r(t) \)).
    \item \texttt{runge\_sis\_}: Función externa que aplica el método de Runge-Kutta de cuarto orden para resolver el sistema asociado a una EDO.
    \item \texttt{y11}, \texttt{y22}: Soluciones del IVP con distintas pendientes iniciales.
    \item \texttt{ys1}: Nueva pendiente calculada mediante el método de la secante.
    \item \texttt{ff}: Función que define la EDO como \( y'' = f(t, y, y') \), y que se implementa por separado en \texttt{ff.m}.
\end{itemize}

\textbf{Programa en Matlab/Octave de la función ff.m:}
\begin{matlabcode}
% AQUI PONGA LA FUNCION DE y' = ay + r(t)
% por ejemplo:   y'' +3y' + 2y = t
% cambio de variable   y1 = y',  y1' = -2*y -3*y1 + t * k
%   x1=@(t,x,y)    y;
%   y1=@(t,x,y) -2*y - 3* y1 + t * k;
%
function yy=ff(t, y , y1 , k)  % k para r(t) <> 0   o  r(t)=0
   yy =  -2*y - 3.*y1 + t;
end
\end{matlabcode}
\textbf{Programa en Matlab/Octave de Disparo\_No\_lineal.m:}
\begin{matlabcode}
% METODO DEL DISPARO NO LINEAL
% por ejemplo:   y'' +0y' + y^2/100 = t;  y(a)=ya ,  y(b)=yb;
% cambio de variable   y1 = y',  y1' = -2*y -3*y1 + t * k
%   x1=@(t,x,y)    y;
%   y1=@(t,x,y) x*x/10  + t ;
%
% % llama a funtion yy=runge_sis_(a,b,x,y,n,k)
%
%  y(t)  =  y1(t) + (yb - y1(b))*y2(t)/y2(b) ,  es la solucion
%
clear all;
a=0; b=1; n=8;  h=(b-a)/n;  err1=.00000001;
x=0; y=01/2;    B=y;% fronteras
% ingresar pendiente 1  y 2
y1=.3; y2=4; %y11=y22=[ ];
k=1;  %   pendiente yn
d=runge_sis_(a,b,x,y1,n,k);
y11=d(1,:);

for i=1: 8000


   % para y2(t)   no homogeneo  r(t) <> 0
   % pendiente yn1
   d=runge_sis_(a,b,x,y2,n,k);
   y22=d(1,:);
   % SOLUCION
   ys1 = y2 - (y22(n+1) - B)/(y22(n+1)-y11(n+1))*(y2-y1);
   printf('%d  yn=%f  yn+1 =%f y(b)=%f \n',i , y2, ys1, y22(n+1));
   if abs(ys1-y2)<err1 break; end;
   y1=y2; y2=ys1; y11=y22;
end; ys1, d(1, n+1)
i, y22
t=a:h:b;
plot(t,y22);
grid;
\end{matlabcode}

\textbf{Manual:}

1. Guarde los archivos como \texttt{ff.m} y \texttt{Disparo\_No\_lineal.m}.

2. Asegúrese de tener la función \texttt{runge\_sis\_} correctamente implementada y disponible en el mismo directorio.

3. Ejecute el programa principal con:
\begin{verbatim}
>> Disparo_No_lineal
\end{verbatim}

4. El script ajusta la pendiente inicial para que la solución del problema de valor inicial cumpla con la condición de frontera en \( t = b \).

5. Al finalizar, se grafica la solución aproximada.

\textbf{Corrida del Programa:}
\begin{matlaboutput}
1  yn=4.000000  yn+1 =1.788797 y(b)=1.014180 
2  yn=1.788797  yn+1 =1.788797 y(b)=0.500000 
ys1 = 1.7888
ans = 0.5000
i = 2
y22 =

        0   0.1858   0.3103   0.3911   0.4414   0.4711   0.4874   0.4957   0.5000
\end{matlaboutput}
\begin{figure}[ht]
\centering
\includegraphics[width=0.5\textwidth]{im6_2.pdf}
\end{figure}

\subsection{Adams-Bashforth de orden 3}
El método Crank-Nicolson es un esquema numérico implícito para resolver ecuaciones en derivadas parciales (EDP) de tipo parabolico, como la ecuación de difusión o calor. Es una combinación de los métodos explícito e implícito, usando el promedio entre ambos en cada paso temporal. Para $\theta=1/2$, se obtiene el método clásico de Crank-Nicolson, que es estable y de segundo orden en el tiempo y espacio.

\textbf{Variables:}

\begin{itemize}
    \item $a$, $b$: Extremos del intervalo espacial.
    \item $n$: Número de subintervalos espaciales.
    \item $h$: Tamaño de paso espacial, $h=(b-a)/n$.
    \item $tt$: Tiempo final de la simulación.
    \item $k$: Tamaño de paso temporal.
    \item $\alpha$: Parámetro de difusión ($\alpha^2$).
    \item $L$: Número de Fourier, $L=\alpha k/h^2$.
    \item $x$: Vector de nodos espaciales.
    \item $w0$: Condición inicial en los nodos.
    \item $A$, $B$: Matrices del sistema lineal para el método.
    \item $BB$: Matriz que almacena la solución en cada paso temporal.
    \item $m$: Número de pasos temporales.
\end{itemize}

\textbf{Programa en Matlab/Octave:}
\begin{matlabcode}
%Crank_nicolson general para theta=1/2
a=0;b=1;n=10;h=(b-a)/n;
tt=1;k=1/12; alp=1; %alp: alpha cuadrado
L=alp*k/h^2;BB=A=B=[];bb=[];
x=a:h:b;w0=x-x.^2; BB(:,1)=w0;
w0=w0(2:n)';
A(1,1)=1+L;A(1,2)=-L/2;
B(1,1)=1-L;B(1,2)=L/2;
for i=2:n-2
    A(i,i-1)=-L/2;A(i,i)=1+L;A(i,i+1)=-L/2;
    B(i,i-1)=L/2;B(i,i)=1-L;B(i,i+1)=L/2;
end
i=n-1;
A(i,i-1)=-L/2;A(i,i)=1+L;
B(i,i-1)=L/2;B(i,i)=1-L;
m=ceil(tt/k);
for i=1:m
    wx=B*w0;wx=A\wx;BB(:,i+1)=[0 wx' 0]';
    w0=wx;
end
BB
mesh(BB)
\end{matlabcode}

\textbf{Manual:}

Para ejecutar el programa, siga estos pasos:

1. Copie el código Matlab/Octave en un archivo llamado `crank\_nicolson.m`.
2. Abra Matlab o GNU Octave y navegue hasta el directorio donde guardó el archivo.
3. Ejecute el archivo escribiendo `crank\_nicolson` en la consola.
4. El resultado será la matriz `BB`, que contiene la solución numérica en cada paso temporal, y una gráfica de la evolución de la solución.

La matriz `BB` muestra cómo la condición inicial evoluciona en el tiempo bajo el método Crank-Nicolson. La gráfica generada (`mesh(BB)`) permite visualizar la difusión de la solución en el dominio espacio-tiempo.
\textbf{Corrida del Programa:}
\begin{matlaboutput}
BB =

 Columns 1 through 8:

        0        0        0        0        0        0        0        0
   0.0900   0.0272   0.0181   0.0033   0.0043  -0.0002   0.0014  -0.0005
   0.1600   0.0594   0.0290   0.0102   0.0052   0.0019   0.0008   0.0005
   0.2100   0.0875   0.0368   0.0158   0.0062   0.0030   0.0009   0.0007
   0.2400   0.1061   0.0419   0.0191   0.0073   0.0034   0.0013   0.0006
   0.2500   0.1126   0.0437   0.0201   0.0077   0.0035   0.0015   0.0005
   0.2400   0.1061   0.0419   0.0191   0.0073   0.0034   0.0013   0.0006
   0.2100   0.0875   0.0368   0.0158   0.0062   0.0030   0.0009   0.0007
   0.1600   0.0594   0.0290   0.0102   0.0052   0.0019   0.0008   0.0005
   0.0900   0.0272   0.0181   0.0033   0.0043  -0.0002   0.0014  -0.0005
        0        0        0        0        0        0        0        0

 Columns 9 through 13:

        0        0        0        0        0
   0.0006  -0.0004   0.0003  -0.0003   0.0002
   0.0000   0.0002  -0.0001   0.0001  -0.0001
   0.0000   0.0002  -0.0001   0.0001  -0.0000
   0.0002   0.0001   0.0001  -0.0000   0.0000
   0.0003   0.0000   0.0001  -0.0000   0.0000
   0.0002   0.0001   0.0001  -0.0000   0.0000
   0.0000   0.0002  -0.0001   0.0001  -0.0000
   0.0000   0.0002  -0.0001   0.0001  -0.0001
   0.0006  -0.0004   0.0003  -0.0003   0.0002
        0        0        0        0        0
\end{matlaboutput}
\begin{figure}[ht]
    \centering
    \includegraphics[width=0.8\textwidth]{im12.pdf}
    \caption{Solución de la EDP con Crank-Nicolson}
    \label{fig:solucion_edp12}
\end{figure}

\subsection{Adams-Bashforth de orden 4}
El método Crank-Nicolson es un esquema numérico implícito utilizado para resolver ecuaciones diferenciales parciales, especialmente la ecuación de difusión o calor. Es conocido por ser estable y tener buena precisión, ya que combina los métodos de Euler hacia adelante y hacia atrás.

\textbf{Variables:}

\begin{itemize}
    \item $L$: Parámetro relacionado con el coeficiente de difusión y el tamaño de la malla.
    \item $A$: Matriz de coeficientes del sistema lineal generado por el método Crank-Nicolson.
    \item $CI$: Vector de condiciones iniciales.
    \item $w0$: Vector de valores de la solución en el paso anterior.
    \item $w1$: Vector de valores de la solución en el paso actual.
    \item $BB$: Matriz que almacena la evolución de la solución en cada paso de tiempo.
    \item $cc$: Vector auxiliar para guardar los resultados en cada iteración.
\end{itemize}

\textbf{Programa en Matlab/Octave:}
\begin{matlabcode}
% Usando Crank -Nicolson 10/07/2025
L = 0.02086;
A = [1-2*L, L, 0, 0;
     L, 1-2*L, L, 0;
     0, L, 1-2*L, L;
     0, 0, L, 1-2*L];

CI = [100*L; 0; 0; 50*L];
w0 = [0; 0; 0; 0];
BB(:,1) = [100; 0; 0; 0; 0; 50];

for i = 1:10
	w1 = A * w0 + CI;
	disp([100, w1', 50]);
	cc = [100, w1', 50]; % Asignar cc para guardarlo
	BB(:,i+1) = cc';
	w0 = w1;
end

surf(BB)
\end{matlabcode}

\textbf{Manual:}

El programa implementa el método de Crank-Nicolson para resolver un problema de difusión en una malla de cuatro nodos. Los pasos para ejecutar el programa son los siguientes:

\begin{enumerate}
    \item Definir el parámetro $L$ según el coeficiente de difusión y el tamaño de la malla.
    \item Construir la matriz $A$ que representa el sistema lineal generado por el método.
    \item Establecer el vector de condiciones iniciales $CI$ y el vector inicial $w0$.
    \item Inicializar la matriz $BB$ para almacenar la evolución temporal de la solución.
    \item Ejecutar el ciclo \texttt{for} para calcular la solución en cada paso de tiempo, actualizando los vectores y almacenando los resultados.
    \item Visualizar la evolución de la solución usando la función \texttt{surf}.
\end{enumerate}

Para modificar el problema, se pueden cambiar los valores de $L$, las condiciones iniciales, el número de nodos o el número de pasos de tiempo. El resultado se muestra tanto en la consola como en una gráfica tridimensional.
\textbf{Corrida del Programa:}
\begin{matlaboutput}
          100        2.086            0            0        1.043           50
          100        4.085     0.043514     0.021757       2.0425           50
          100       6.0015      0.12736     0.064363       3.0007           50
          100       7.8397      0.24858      0.12693       3.9199           50
          100       9.6038       0.4044      0.20859        4.802           50
          100       11.298      0.59221      0.30849        5.649           50
          100       12.925      0.80961      0.42581       6.4628           50
          100       14.488       1.0543      0.55975        7.245           50
          100       15.992       1.3242      0.70952       7.9974           50
          100       17.438       1.6174      0.87437       8.7216           50
\end{matlaboutput}
\begin{figure}[ht]
    \centering
    \includegraphics[width=0.8\textwidth]{media/im13.pdf}
    \caption{Gráfica de la corrida del programa}
    \label{fig:informe13}
\end{figure}

\section{Método de Moulton}
Es un ejemplo de uso del método de Moulton para resolver la ecuación diferencial ordinaria (EDO):
\begin{equation*}
    y'=t-ty \quad y(0)=\frac{1}{2}\quad t\in[0,1]
\end{equation*}
Este método es un método implícito de predicción-corrección que utiliza el valor de la función en el paso siguiente para corregir el valor actual. Por ese motivo, se utiliza la función \texttt{runge11.m} que se definió en el informe 8.

Sea crea un nuevo programa con el nombre de \textbf{moulton.m}.

\textbf{Código en Matlab/Octave:}
\begin{matlabcode}
    clear all;
    y0=1/2; %condición inicial
    a=0;b=1;% intervalo
    n=4; % número de pasos
    %----------------------
    t=a;
    h=(b-a)/n;
    yt=[];yt(1)=y0;
    u=runge11(a,b,h,1,y0);
    yt=u;
    t=a+h;
    y0=yt(2);
    for i=2:n
      v=y0;
      for k=1:5
        v1=y0+h/12*(5*ff(t+h,v)+8*ff(t,yt(i))-ff(t-h,yt(i-1)));
        if abs(v1-v)<0.0001 
            break
        end
        v=v1;
      end%k
      y0=v1;t=t+h;yt(i+1)=y0;
    end%i
    u=a:h:b;
    disp([u',yt']);
\end{matlabcode}
\textbf{Manual:}
\begin{itemize}
    \item Abrir el archivo \textbf{moulton.m} en Matlab/Octave.
    \item En el programa \texttt{ff.m} se define la EDO a resolver.
    \item En las lineas 
    \begin{verbatim}
    y0=1/2; %condición inicial
    a=0;b=1;% intervalo
    n=4; % número de paso
    \end{verbatim}
    se definen los parámetros de la EDO a resolver
    \item En la línea \textbf{n=4;} se define el número de pasos a realizar.
    \item Ejecutar el script con el comando \textbf{moulton} en la consola de Matlab/Octave.
\end{itemize}
\textbf{Corrida del Programa:}
\begin{matlaboutput}
>> Moulton
    0    0.5000
0.2500    0.5154
0.5000    0.5586
0.7500    0.6223
1.0000    0.6965
\end{matlaboutput}

\subsection{Método de Moulton de orden 3}
Es un ejemplo de uso del método de Moulton de orden 3 para resolver la ecuación diferencial ordinaria (EDO):
\begin{equation*}
    y'=t-ty \quad y(0)=\frac{1}{2}\quad t\in[0,1]
\end{equation*}
Este método es un método implícito de predicción-corrección que utiliza el valor de la función en el paso siguiente para corregir el valor actual. Por ese motivo, se utiliza la función \texttt{runge11.m} que se definió en el informe 8.

Sea crea un nuevo programa con el nombre de \textbf{moulton3.m}.

\textbf{Código en Matlab/Octave:}
\begin{matlabcode}
    clear all;
    y0=1/2;%condición inicial
    a=0;b=1;% intervalo
    n=4;% número de pasos
    %--------------------
    t=a;
    h=(b-a)/n;
    yt=[];yt(1)=y0;
    u=runge11(a,b,h,2,y0);
    yt=u;
    t=a+2*h;
    y0=yt(3);
    for i=3:n
      v=y0;
      for k=1:5
        v1=y0+h/24*(9*ff(t+h,v)+19*ff(t,yt(i))-5*ff(t-h,yt(i-1))+ff(t-2*h,yt(i-2)));
        if abs(v1-v)<0.0001 
            break
        end
        v=v1;
      end%k
      y0=v1;t=t+h;yt(i+1)=y0;
    end%i
    u=a:h:b;
    disp([u',yt']);
\end{matlabcode}
\textbf{Manual:}
\begin{itemize}
    \item Abrir el archivo \textbf{moulton3.m} en Matlab/Octave.
    \item En el programa \texttt{ff.m} se define la EDO a resolver.
    \item En las lineas 
    \begin{verbatim}
    y0=1/2; %condición inicial
    a=0;b=1;% intervalo
    n=4; % número de paso
    \end{verbatim}
    se definen los parámetros de la EDO a resolver
    \item En la línea \textbf{n=4;} se define el número de pasos a realizar.
    \item Ejecutar el script con el comando \textbf{moulton3} en la consola de Matlab/Octave.
\end{itemize}
\textbf{Corrida del Programa:}
\begin{matlaboutput}
>> Moulton3
    0    0.5000
0.2500    0.5154
0.5000    0.5588
0.7500    0.6226
1.0000    0.6968
\end{matlaboutput}

\section{Graficar sistemas de ecuaciones diferenciales(EDO) de orden 1}
Es un program que gráfica las soluciones de un sistema de EDO's de orden 1:
\begin{equation*}
    \begin{cases}
        x' = 2x + 3y \\
        y' = \frac{2}{3}x + 3y
    \end{cases}
\quad x(0)=0, y(0)=1 \quad t\in[0,1]
\end{equation*}
Sea crea un nuevo programa con el nombre de \textbf{runge4sistema.m}.

\textbf{Código en Matlab/Octave:}
\begin{matlabcode}
    format long G
    t=0:0.01:1;%dominio
    x=@(t) exp(4*t)-exp(t); %soluciones
    y=@(t) 2/3*exp(4*t)+1/3*exp(t); %soluciones
    plot(t,x(t),'r',t,y(t),'b')
    u=0:1/2:1;
    [u' x(u)' y(u)']
\end{matlabcode}
\textbf{Manual:}
\begin{itemize}
    \item Abrir el archivo \textbf{runge4sistema.m} en Matlab/Octave.
    \item Ejecutar el script con el comando \textbf{runge4sistema} en la consola de Matlab/Octave.
\end{itemize}
\textbf{Corrida del Programa:}
\begin{matlaboutput}
    ans =

    0                         0                         1
  0.5          5.74033482823052          5.47561115618714
    1          51.8798682046852          37.3048606315825
\end{matlaboutput}
\begin{figure}[ht]
    \centering
    \includegraphics[width = 10cm]{grafica17.pdf}
\end{figure}

\section{Disparo lineal}
Es un \textit{script} que calcula las cotas del error del método de Euler para una ecuación diferencial ordinaria (EDO) dada. El método de Euler es un método numérico para resolver EDOs, y las cotas del error proporcionan una estimación de la precisión de la solución aproximada en comparación con la solución exacta.
\begin{theo}{}{}
Suponga que \textit{f} es continua y satisface la condición de Lipschitz con constante $L$ en 
\begin{equation*}
D=\{(t,y)|a\leq t\leq b \text{ y }-\infty<y<\infty\}
\end{equation*}
y que existe una constante $M$ con
\begin{equation*}
|y''(t)|\leq M,\text{ para todas las }t\in[a,b],
\end{equation*}
donde $y(t)$ denota la única solución para el problema de valor inicial
\begin{equation*}
y'=f(t,y),\quad a\leq t\leq b,\quad y(a)=\alpha.
\end{equation*}
Sean $w_0,w_1,...,w_N$ las aproximaciones generadas por el método de Euler para un entero positivo $N$. Entonces, para cada $i=0,1,2,\cdots,N,$
\begin{equation}
|y(t_i)-w_i|\leq\frac{hM}{2L}\left[e^{L(t_i-a)}-1\right]
\end{equation}
\end{theo}
\textbf{Ejemplo:} Encontrar las cotas del error de Euler para la ecuación diferencial
\begin{equation*}
y' = t - y, \quad y(0) = 0 \text{ con } 0 \leq t \leq 1
\end{equation*}
\textbf{Solución:}

La solución analítica de la ecuación diferencial es
\begin{equation*}
y = t - 1 + e^{-t}
\end{equation*}
y se quiere comparar con la solución aproximada.

Creamos un script llamado \textbf{cotas\_euler.m} que llama a la función \textbf{euler.m} de la sección anterior.

\begin{matlabcode}
%a,b: intervalo donde se define la EDO
%M : constante
%N : numero de pasos para h=(b-a)/N
%L : constante de Lipschitz
%f : f(x,y) de y'=f(x,y)
%y : solucion analitica del problema de valor inicial
%y0=alpha : condicion inicial
a=0; b=1; %t en [a,b]
f = @(t,y) t-y; %funcion f(t,y) de la EDO
M=1;
L=1;
N=3; %para h=1/3
y = @(t) t-1+exp(-t); %sol. analitica de la EDO
y0=0; %condicion inicial
%-----------------------------
% Cotas del error de Euler
%-----------------------------
h=(b-a)/N;
[t,w]=euler(a,b,N,y0,f);
fprintf('%8s %15s %20s\n', 't_i', '|y_i - w_i|', 'cota de error');
for i = 1:N
    ti = t(i);
    err = abs(y(ti) - w(i));
    bound = (h * M / (2 * L)) * abs(exp(L * (ti - a)) - 1);
    fprintf('%8.4f %15.6f %20.6f\n', ti, err, bound);
end
\end{matlabcode}
\textbf{Resultados:}
\begin{matlaboutput}
>> cotas_euler
     t_i     |y_i - w_i|        cota de error
  0.0000        0.000000             0.000000
  0.3333        0.049865             0.065935
  0.6667        0.068973             0.157956
\end{matlaboutput}

\subsection{Método del disparo con Runge-Kutta de orden 4}
Este programa implementa el método del disparo para resolver un problema de valor en la frontera (boundary value problem, BVP) para un sistema de ecuaciones diferenciales de primer orden. El método convierte el BVP en un problema de valor inicial (IVP), que luego se resuelve usando el método de Runge-Kutta de orden 4.

En particular, el sistema resuelto es:
\[
\begin{cases}
x' = 2x + 3y \\
y' = \frac{2}{3}x + 3y \\
x(0) = 0, \quad y(0) = 1, \quad t \in [0,1]
\end{cases}
\]

Este tipo de técnica es común cuando se conocen condiciones en los extremos del intervalo y se requiere una solución aproximada de alta precisión.

---

\textbf{Variables:}

\begin{itemize}
    \item \texttt{a}, \texttt{b}: Extremos del intervalo de integración en el tiempo (\( t \in [a, b] \)).
    \item \texttt{x}, \texttt{y}: Condiciones iniciales \( x(a) \), \( y(a) \).
    \item \texttt{n}: Número de subintervalos en los que se divide el intervalo \([a, b]\); define la precisión del método.
    \item \texttt{h}: Tamaño del paso \( h = \frac{b - a}{n} \).
    \item \texttt{t}: Variable del tiempo que avanza en cada iteración.
    \item \texttt{x0}, \texttt{y0}: Valores actuales de las variables \( x \) e \( y \) en cada paso.
    \item \texttt{xt}, \texttt{yt}: Vectores donde se almacenan los valores aproximados de \( x(t) \) y \( y(t) \).
    \item \texttt{kij}: Coeficientes intermedios del método de Runge-Kutta para calcular la solución numérica, tanto para \( x \) como para \( y \).
    \item \texttt{fff(t,x,y)}: Función que representa la derivada de \( x \) en el sistema (es decir, \( x' = f(x,y) \)).
    \item \texttt{fffg(t,x,y)}: Función que representa la derivada de \( y \) en el sistema (es decir, \( y' = g(x,y) \)).
    \item \texttt{yy}: Matriz de dos filas: la primera contiene los valores de \( x(t) \), y la segunda los de \( y(t) \).
\end{itemize}

---


\textbf{Programa en Matlab/Octave:}
\begin{matlabcode}
% x' = 2x   + 3y
% y' = 2x/3 + 3y ;       x(0)=0;  y(0)=1;  t en [0,1]
function yy=disparo_r_k_11(a,b,x,y,n)
h=(b-a)/n; t=a;
x0=x; y0=y; xt(1)=x0; yt(1)=y0;
% k11 k12 k13 k14 para fff(t)
% k21  k22  k23 k24 para fffg(t)
for i=1:n
  k11 = h*fff(t, x0,y0);                  k21=h*fffg(t,x0,y0);
  k12 = h*fff(t+h/2, x0+k11/2,y0+k21/2);  k22=h*fffg(t+h/2,x0+k11/2,y0+k21/2);
  k13 = h*fff(t+h/2, x0+k12/2,y0+k22/2);  k23=h*fffg(t+h/2,x0+k12/2,y0+k22/2);
  k14 = h*fff(t+h, x0+k13,y0+k23);        k24=h*fffg(t+h,x0+k13,y0+k23);
  xs=x0+(k11 + 2*(k12+k13) + k14)/6; xt(i+1)=xs;
  ys=y0+(k21 + 2*(k22+k23) + k24)/6; yt(i+1)=ys;
  t=t+h; x0=xs; y0=ys;
end;
yy=[xt ; yt];

\end{matlabcode}

\textbf{Manual:}
Se escribe un nuevo script de nombre disparo\_r\_k\_11.m, que es el programa que resuelve el problema de valor de frontera con el método de Runge-Kutta de orden 4. Este programa usa las funciones fff(t,x,y) y fffg(t,x,y) que se escribieron en los informes siguientes.

\textbf{Corrida del Programa:}
Este programa no se ejecuta directamente, sino que se llama desde otro programa de nombre llamador.m.

\subsection{Método del disparo con fórmula de las secantes}
Es un método numérico para resolver ecuaciones diferenciales ordinarias (EDO) de la forma
\begin{equation*}
    y' = f(t,y), \quad y(t_0) = y_0\quad  a\leq t\leq b
\end{equation*}
donde $f(t,y)$ es una función continua y $y_0$ es el valor inicial de la solución en el punto $t_0$. El método Runge-Kutta de orden 3 es una extensión del método de Runge-Kutta de orden 2, proporcionando una mayor precisión en la aproximación de la solución.

Para $n=3$, es posible efectuar un desarrollo similar al del método de segundo orden. Una versión comun que se obtiene es
\begin{equation}
    y_{i+1}=y_i+\frac{1}{6}(k_1+4k_2+k_3)h
\end{equation}
donde
\begin{align*}
    k_1&=f(x_i,y_i)\\
    k_2&=f(x_i+\frac{1}{2}h,y_i+\frac{1}{2}k_1h)\\
    k_3&=f(x_i+h,y_i-k_1h+2k_2h)
\end{align*}
Sea la EDO que no se puede resolver analiticamente
\begin{equation*}
    y'=t-ty^{1.5}
\end{equation*}
Entonces utilizando Runge-Kutta de orden 3, creamos un archivo en matlab/octave llamado RKorden3.m, y escribimos el siguiente codigo.

\textbf{Código en Matlab/Octave:}
\begin{matlabcode}
    %Runge Kutta de orden 3
    y1 = @(t,y) t-y^(3/2); %Ec. Diff. Ordinaria a solucionar
    y0=1; %condición inicial
    a=1;b=2; %Intervalos
    N=8;
    h = (b-a)/N;
    yt=zeros(1,N);yt(1)=y0;
    a1=a;
    for i=1:N
        k1=h*y1(a1,y0);
        k2=h*y1(a1+h/2,y0+k1/2);
        k3=h*y1(a1+h,y0+k1+2*k2);
        yp=y0+(k1+4*k2+k3)/6;yt(i+1)=yp;
        fprintf('%d| yn=%f| k1=%f| k2=%f| k3=%f y_n+1=%f\n',i,y0,k1,k2,k3,yp)
        y0=yp; a1=a1+h;
    end
      t=a:h:b;
      pa=polyfit(t,yt,4); % minimos cuadrados
      pt=polyval(pa,t);
      plot(t,pt,t,yt,'.r','MarkerSize',10)
\end{matlabcode}
\textbf{Manual:}
\begin{itemize}
    \item Abrir el archivo \textbf{Rkorden3.m} en Matlab/Octave.
    \item En las lineas 
    \begin{verbatim}
y1 = @(t,y) t-y^(3/2); %Ec. Diff. Ordinaria a solucionar
y0=1; %condición inicial
a=1;b=2; %Intervalos
    \end{verbatim}
    Se ingresa la ecuación diferencial ordinaria a solucionar, la condición inicial y el intervalo de solución.
    \item En la línea \textbf{N=8;} se define el número de pasos a realizar.
    \item Ejecutar el script con el comando \textbf{Rkorden3} en la consola de Matlab/Octave.
\end{itemize}
\textbf{Corrida del Programa:}
\begin{matlaboutput}
    >> RKorden3
    1| yn=1.000000| k1=0.000000| k2=0.007812| k3=0.012684 y_n+1=1.007322
    2| yn=1.007322| k1=0.014250| k2=0.020719| k3=0.019251 y_n+1=1.026718
    3| yn=1.026718| k1=0.026207| k2=0.031522| k3=0.024512 y_n+1=1.056186
    4| yn=1.056186| k1=0.036193| k2=0.040504| k3=0.028619 y_n+1=1.093991
    5| yn=1.093991| k1=0.044469| k2=0.047899| k3=0.031722 y_n+1=1.138622
    6| yn=1.138622| k1=0.051252| k2=0.053909| k3=0.033965 y_n+1=1.188764
    7| yn=1.188764| k1=0.056736| k2=0.058714| k3=0.035482 y_n+1=1.243277
    8| yn=1.243277| k1=0.061090| k2=0.062477| k3=0.036398 y_n+1=1.301176
\end{matlaboutput}
\begin{figure}[ht]
    \centering
    \includegraphics[width = 10cm]{figure2_info5.pdf}
\end{figure}

\subsection{Método del disparo para ecuaciones diferenciales no lineales}
Este conjunto de programas resuelve un problema de valor en la frontera no lineal (BVP) de la forma:

\[
y'' + 3y' + 2y = t, \quad y(a)=ya, \quad y(b)=yb,
\]

utilizando el método del disparo, que transforma el problema BVP en un problema de valor inicial (IVP).  
La idea es proponer una pendiente inicial \( y'(a) = s \), resolver el sistema con esta condición inicial mediante el método de Runge-Kutta, y ajustar \( s \) iterativamente hasta que la solución cumpla con la condición de frontera en \( b \).

Se implementa una versión no lineal del método del disparo, en la cual se aplica el método de la secante para aproximar la pendiente correcta que satisface la condición final. La no linealidad del problema requiere un ajuste iterativo de esta pendiente hasta obtener el valor deseado en \( y(b) \).

---

\textbf{Variables:}

\begin{itemize}
    \item \texttt{a}, \texttt{b}: Extremos del intervalo del dominio de la solución.
    \item \texttt{n}: Número de subintervalos para la discretización del intervalo.
    \item \texttt{h}: Tamaño del paso, calculado como \( h = (b-a)/n \).
    \item \texttt{y1}, \texttt{y2}: Pendientes iniciales propuestas para el disparo, que se actualizan con el método de la secante.
    \item \texttt{k}: Parámetro que representa el control de la no homogeneidad del sistema (presencia del término \( r(t) \)).
    \item \texttt{runge\_sis\_}: Función externa que aplica el método de Runge-Kutta de cuarto orden para resolver el sistema asociado a una EDO.
    \item \texttt{y11}, \texttt{y22}: Soluciones del IVP con distintas pendientes iniciales.
    \item \texttt{ys1}: Nueva pendiente calculada mediante el método de la secante.
    \item \texttt{ff}: Función que define la EDO como \( y'' = f(t, y, y') \), y que se implementa por separado en \texttt{ff.m}.
\end{itemize}

\textbf{Programa en Matlab/Octave de la función ff.m:}
\begin{matlabcode}
% AQUI PONGA LA FUNCION DE y' = ay + r(t)
% por ejemplo:   y'' +3y' + 2y = t
% cambio de variable   y1 = y',  y1' = -2*y -3*y1 + t * k
%   x1=@(t,x,y)    y;
%   y1=@(t,x,y) -2*y - 3* y1 + t * k;
%
function yy=ff(t, y , y1 , k)  % k para r(t) <> 0   o  r(t)=0
   yy =  -2*y - 3.*y1 + t;
end
\end{matlabcode}
\textbf{Programa en Matlab/Octave de Disparo\_No\_lineal.m:}
\begin{matlabcode}
% METODO DEL DISPARO NO LINEAL
% por ejemplo:   y'' +0y' + y^2/100 = t;  y(a)=ya ,  y(b)=yb;
% cambio de variable   y1 = y',  y1' = -2*y -3*y1 + t * k
%   x1=@(t,x,y)    y;
%   y1=@(t,x,y) x*x/10  + t ;
%
% % llama a funtion yy=runge_sis_(a,b,x,y,n,k)
%
%  y(t)  =  y1(t) + (yb - y1(b))*y2(t)/y2(b) ,  es la solucion
%
clear all;
a=0; b=1; n=8;  h=(b-a)/n;  err1=.00000001;
x=0; y=01/2;    B=y;% fronteras
% ingresar pendiente 1  y 2
y1=.3; y2=4; %y11=y22=[ ];
k=1;  %   pendiente yn
d=runge_sis_(a,b,x,y1,n,k);
y11=d(1,:);

for i=1: 8000


   % para y2(t)   no homogeneo  r(t) <> 0
   % pendiente yn1
   d=runge_sis_(a,b,x,y2,n,k);
   y22=d(1,:);
   % SOLUCION
   ys1 = y2 - (y22(n+1) - B)/(y22(n+1)-y11(n+1))*(y2-y1);
   printf('%d  yn=%f  yn+1 =%f y(b)=%f \n',i , y2, ys1, y22(n+1));
   if abs(ys1-y2)<err1 break; end;
   y1=y2; y2=ys1; y11=y22;
end; ys1, d(1, n+1)
i, y22
t=a:h:b;
plot(t,y22);
grid;
\end{matlabcode}

\textbf{Manual:}

1. Guarde los archivos como \texttt{ff.m} y \texttt{Disparo\_No\_lineal.m}.

2. Asegúrese de tener la función \texttt{runge\_sis\_} correctamente implementada y disponible en el mismo directorio.

3. Ejecute el programa principal con:
\begin{verbatim}
>> Disparo_No_lineal
\end{verbatim}

4. El script ajusta la pendiente inicial para que la solución del problema de valor inicial cumpla con la condición de frontera en \( t = b \).

5. Al finalizar, se grafica la solución aproximada.

\textbf{Corrida del Programa:}
\begin{matlaboutput}
1  yn=4.000000  yn+1 =1.788797 y(b)=1.014180 
2  yn=1.788797  yn+1 =1.788797 y(b)=0.500000 
ys1 = 1.7888
ans = 0.5000
i = 2
y22 =

        0   0.1858   0.3103   0.3911   0.4414   0.4711   0.4874   0.4957   0.5000
\end{matlaboutput}
\begin{figure}[ht]
\centering
\includegraphics[width=0.5\textwidth]{im6_2.pdf}
\end{figure}

\section{Método de diferencias finitas}
\subsection{Método de diferencias finitas para problemas lineales}
Este programa resuelve un problema de valor en la frontera (BVP) para una ecuación diferencial lineal de segundo orden utilizando el método de diferencias finitas. En lugar de construir la matriz del sistema de forma automática, en este caso se ha calculado manualmente la matriz de coeficientes \( A \) y el vector del lado derecho \( b \), lo que simplifica la implementación y permite concentrarse en el concepto del método.

Se supone que la EDO tiene la forma general:
\[
y'' + p(x)y' + q(x)y = r(x), \quad y(0) = 0,\ y(1) = 1
\]
y que el dominio \([0,1]\) se ha discretizado con \( n = 4 \) puntos (3 nodos interiores).

---

\textbf{Variables:}

\begin{itemize}
    \item \texttt{A}: Matriz de coeficientes del sistema lineal generado por el método de diferencias finitas, con los valores ya ingresados manualmente.
    \item \texttt{b}: Vector del lado derecho del sistema, calculado según las condiciones de frontera y los valores de la función \( r(x) \) en los nodos internos.
    \item \texttt{y}: Vector solución del sistema lineal \( A \cdot y = b \), que contiene las aproximaciones de la función en los nodos interiores. Se completa con las condiciones de frontera: \( y(0) = 0 \) y \( y(1) = 1 \).
    \item \texttt{x}: Vector que contiene los puntos del dominio equiespaciados en el intervalo \([0,1]\), incluyendo los extremos.
\end{itemize}

\textbf{Programa en Matlab/Octave:}
\begin{matlabcode}
A=[-15/8 11/8 0; 5/8 -15/8 11/8;0 5/8 -15/8];
b=[1/64 1/32 -85/64];
y=inv(A)*b'
y=[0 y' 1];
x=0:1/4:1;
plot(x,y)
\end{matlabcode}

\textbf{Manual:}

1. Guarde el código en un archivo llamado \texttt{dif\_finitas\_manual.m}.

2. Ejecute desde la consola de Octave o Matlab con:
\begin{verbatim}
>> dif_finitas_manual
\end{verbatim}

3. El programa resuelve el sistema lineal ingresado manualmente y grafica la función aproximada \( y(x) \) en el intervalo \([0,1]\), incluyendo los valores de frontera.

4. La matriz \( A \) y el vector \( b \) se han deducido manualmente a partir de la discretización por diferencias finitas y de las condiciones de contorno.

\textbf{Corrida del Programa:}
\begin{matlaboutput}
y =

   0.7091
   0.9783
   1.0344

\end{matlaboutput}
\begin{figure}[ht]
\centering
\includegraphics[width=0.5\textwidth]{im2_2.pdf}
\end{figure}

\subsection{Método de diferencias finitas con contrucción automática de matriz}
Este programa resuelve un problema de valor en la frontera (BVP) para una ecuación diferencial de segundo orden utilizando el método de diferencias finitas, donde la matriz de coeficientes y el vector del segundo miembro se generan automáticamente dentro del código.

El problema considerado es de la forma:
\[
y'' + p y' + q y = r(x), \quad \text{en } [a,b], \quad y(a) = y_a, \ y(b) = y_b
\]
En este caso particular:

- \( p = 3 \), \( q = 2 \)

- \( r(x) = x \) (función lineal)

- Condiciones de frontera: \( y(0) = 0 \), \( y(1) = 1 \)

Se discretiza el intervalo \([0,1]\) en \(n = 10\) subintervalos y se construye un sistema lineal \( A \cdot y = b \), que se resuelve para encontrar las aproximaciones de \( y(x) \) en los nodos interiores.

---

\textbf{Variables:}

\begin{itemize}
    \item \texttt{p}, \texttt{q}: Coeficientes de la EDO en los términos \( y' \) y \( y \) respectivamente.
    \item \texttt{r(x)}: Función fuente del lado derecho de la ecuación. En este caso, es \( r(x) = x \).
    \item \texttt{a}, \texttt{b}: Extremos del intervalo donde se resuelve el problema.
    \item \texttt{n}: Número de subintervalos de la discretización (hay \(n-1\) nodos interiores).
    \item \texttt{ya}, \texttt{yb}: Valores de las condiciones de frontera en \(x = a\) y \(x = b\), respectivamente.
    \item \texttt{h}: Paso de la malla, \( h = \frac{b-a}{n} \).
    \item \texttt{A}: Matriz de coeficientes del sistema lineal construida con base en la discretización de la EDO.
    \item \texttt{b}: Vector del segundo miembro construido en función de \( r(x) \) y de las condiciones de frontera.
    \item \texttt{cc}: Solución del sistema lineal, concatenada con los valores de frontera para obtener la solución completa.
    \item \texttt{t}: Vector que contiene los nodos de la discretización.
\end{itemize}


\textbf{Programa en Matlab/Octave:}
\begin{matlabcode}
% y''+py''+qy=r(x)
p=3;q=2;a=0;b=1;n=10;
ya=0;yb=1;
%------------------
h=(b-a)/n;
A=[];b1=b;
A(1,1)=2*h.^2-2;
A(1,2)=1+3*h/2;
b(1)=(a+h)*h.^2-(1-3*h/2)*ya;
for i=2:(n-2)
        A(i,i-1)=1-3*h/2;
        A(i,i)=2*h*h-2;
        A(i,i+1)=1+3*h/2;
        b(i)=h.^2*(i*h);
end%i
A(n-1,n-2)=1-3*h/2;A(n-1,n-1)=2*h.^2-2;b(n-1)=(a+(n-1)*h)*h.^2-(1+3*h/2)*yb;
[A b']
cc=A\b';cc=[ya cc' yb];
t=a:h:b1;
plot(t,cc,'LineWidth', 2);  % Cambia el 2 por el grosor que desees
\end{matlabcode}

\textbf{Manual:}

1. Guarde el código en un archivo llamado \texttt{dif\_finitas\_auto.m}.

2. Ejecute en Octave o Matlab con:
\begin{verbatim}
>> dif_finitas_auto
\end{verbatim}

3. El programa construye automáticamente la matriz \( A \) y el vector \( b \) a partir de los coeficientes \( p \), \( q \), y de la función \( r(x) = x \).

4. Resuelve el sistema \( A \cdot y = b \) e incorpora las condiciones de frontera para graficar la solución aproximada.

\textbf{Corrida del Programa:}
\begin{matlaboutput}
ans =

  -1.9800   1.1500        0        0        0        0        0        0        0   0.0010
   0.8500  -1.9800   1.1500        0        0        0        0        0        0   0.0020
        0   0.8500  -1.9800   1.1500        0        0        0        0        0   0.0030
        0        0   0.8500  -1.9800   1.1500        0        0        0        0   0.0040
        0        0        0   0.8500  -1.9800   1.1500        0        0        0   0.0050
        0        0        0        0   0.8500  -1.9800   1.1500        0        0   0.0060
        0        0        0        0        0   0.8500  -1.9800   1.1500        0   0.0070
        0        0        0        0        0        0   0.8500  -1.9800   1.1500   0.0080
        0        0        0        0        0        0        0   0.8500  -1.9800  -1.1410
\end{matlaboutput}
\begin{figure}[ht]
    \centering
    \includegraphics[width=0.5\textwidth]{im3_2.pdf}
\end{figure}

\subsection{Método de diferencias finitas iterativo para problema no lineal}
Este informe presenta la resolución numérica de una ecuación diferencial parcial parabólica utilizando el método de diferencias finitas implícito. El objetivo es aproximar la solución de la EDP en un dominio definido, mostrando el procedimiento, el código implementado y los resultados obtenidos.

\textbf{Variables:}

\begin{itemize}
   \item \texttt{ff}: función fuente del problema, definida como \( f(x) = 2x - x^2 \).
   \item \texttt{a}, \texttt{b}: extremos del intervalo espacial (\( a = 0 \), \( b = 2 \)).
   \item \texttt{n}: número de subintervalos espaciales (\( n = 10 \)).
   \item \texttt{h}: tamaño de paso espacial (\( h = (b-a)/n \)).
   \item \texttt{k}: tamaño de paso temporal (\( k = 1/100 \)).
   \item \texttt{al}: parámetro de difusión (\( \alpha = 4 \)).
   \item \texttt{lam}: parámetro auxiliar (\( \lambda = \alpha k / h^2 \)).
   \item \texttt{A}: matriz de coeficientes del sistema lineal.
   \item \texttt{bb}: vector del lado derecho en cada paso temporal.
   \item \texttt{B}: matriz que almacena las soluciones en cada paso temporal.
   \item \texttt{x}, \texttt{x1}: vectores de nodos espaciales.
   \item \texttt{c}: solución del sistema en cada iteración temporal.
\end{itemize}

\textbf{Programa en Matlab/Octave:}
\begin{matlabcode}
%parabolicas 2025/07/03
%este programa funciona con casi todo
ff=@(x) 2*x-x.^2;
a=0;b=2;n=10;h=(b-a)/n;
k=1/100;al=4;lam=al*k/(h.^2);
A=[];bb=[];B=A;
A(1,1)=1+2*lam;A(1,2)=-lam;
for i=2:n-2
    A(i-1,i)=-lam;A(i,i)=1+2*lam;A(i,i+1)=-lam;
end
i=n-1;A(i,i-1)=-lam;A(i,i)=1+2*lam
x=a:h:b;x1=x(2:n);bb=ff(x1)';B(:,1)=[0 bb' 0];
for i=1:10
    c=A\bb
    B(:,i+1)=[0 c' 0];
    bb=c;
end
mesh(B)
\end{matlabcode}

\textbf{Manual:}

Para resolver la ecuación diferencial parcial parabólica por el método de diferencias finitas implícito, se sigue el siguiente procedimiento:

1. **Discretización del dominio:** Se divide el intervalo espacial \([a, b]\) en \(n\) subintervalos de tamaño \(h\), y el tiempo en pasos de tamaño \(k\).

2. **Formulación de la ecuación discreta:** La ecuación parabólica se aproxima usando diferencias finitas centradas en el espacio y hacia adelante en el tiempo, resultando en un sistema lineal para cada paso temporal.

3. **Construcción de la matriz de coeficientes:** Se arma la matriz tridiagonal \(A\) que representa la relación entre los nodos interiores en cada paso temporal, incorporando el parámetro de difusión \(\alpha\) y el parámetro auxiliar \(\lambda\).

4. **Condiciones iniciales y de frontera:** Se establecen los valores iniciales y las condiciones de frontera en los extremos del dominio.

5. **Resolución iterativa:** En cada paso temporal, se resuelve el sistema lineal \(A c = bb\) para obtener la solución en los nodos interiores, actualizando el vector de solución para el siguiente paso.

6. **Almacenamiento y visualización:** Las soluciones en cada paso temporal se almacenan en la matriz \(B\) y se visualizan mediante una gráfica de malla (mesh).

Este método es estable y adecuado para problemas parabólicos, permitiendo aproximar la evolución temporal de la solución en el dominio definido.
\textbf{Corrida del Programa:}
\begin{matlaboutput}
A =

 Columns 1 through 8:

   3.0000  -1.0000        0        0        0        0        0        0
        0   3.0000  -1.0000        0        0        0        0        0
        0        0   3.0000  -1.0000        0        0        0        0
        0        0        0   3.0000  -1.0000        0        0        0
        0        0        0        0   3.0000  -1.0000        0        0
        0        0        0        0        0   3.0000  -1.0000        0
        0        0        0        0        0        0   3.0000  -1.0000
        0        0        0        0        0        0        0   3.0000
        0        0        0        0        0        0        0  -1.0000

 Column 9:

        0
        0
        0
        0
        0
        0
        0
  -1.0000
   3.0000

c =

   0.2400
   0.3601
   0.4402
   0.4806
   0.4817
   0.4450
   0.3750
   0.2850
   0.2150

c =

   0.1451
   0.1952
   0.2256
   0.2365
   0.2288
   0.2049
   0.1696
   0.1337
   0.1162

c =

   0.082651
   0.102877
   0.113426
   0.114729
   0.107720
   0.094317
   0.078090
   0.064687
   0.060312

c =

   0.045221
   0.053013
   0.056163
   0.055062
   0.050456
   0.043649
   0.036629
   0.031797
   0.030703

c =

   0.024015
   0.026825
   0.027462
   0.026224
   0.023609
   0.020371
   0.017464
   0.015762
   0.015488

c =

   1.2463e-02
   1.3374e-02
   1.3298e-02
   1.2431e-02
   1.1071e-02
   9.6025e-03
   8.4368e-03
   7.8467e-03
   7.7783e-03

c =

   6.3508e-03
   6.5892e-03
   6.3933e-03
   5.8820e-03
   5.2146e-03
   4.5732e-03
   4.1172e-03
   3.9148e-03
   3.8977e-03

c =

   3.1890e-03
   3.2161e-03
   3.0592e-03
   2.7844e-03
   2.4712e-03
   2.1991e-03
   2.0241e-03
   1.9553e-03
   1.9510e-03

c =

   1.5826e-03
   1.5588e-03
   1.4601e-03
   1.3212e-03
   1.1792e-03
   1.0665e-03
   1.0004e-03
   9.7710e-04
   9.7603e-04

c =

   7.7811e-04
   7.5175e-04
   6.9649e-04
   6.2931e-04
   5.6673e-04
   5.2093e-04
   4.9628e-04
   4.8841e-04
   4.8815e-04
\end{matlaboutput}
\begin{figure}[ht]
    \centering
    \includegraphics[width=0.8\textwidth]{im9.pdf}
    \caption{Solución de la EDP}
    \label{fig:solucion_edp9}
\end{figure}

\subsection{Diferencias finitas para EDP's de tipo parabólico}
Este programa implementa el método de diferencias finitas para resolver una ecuación diferencial parcial (EDP) de tipo parabólico, utilizando Matlab/Octave. Se calcula la evolución temporal de la solución en una malla discreta, mostrando los resultados numéricos y la gráfica correspondiente.

\textbf{Variables:} \\
\begin{itemize}
  \item \texttt{la}: Parámetro de estabilidad, calculado como $(4 \times 1/65) / (0.5^2)$.
  \item \texttt{A}: Matriz de coeficientes del sistema lineal generado por el método de diferencias finitas.
  \item \texttt{n}: Número de divisiones espaciales (nodos menos uno).
  \item \texttt{h}: Tamaño de paso espacial, calculado como $(2-0)/n$.
  \item \texttt{x}: Vector de posiciones espaciales.
  \item \texttt{w0}: Vector de condiciones iniciales en los nodos internos.
  \item \texttt{B}: Matriz que almacena la evolución de la solución en cada paso temporal.
  \item \texttt{w1}: Vector de solución en el siguiente paso temporal.
\end{itemize}

\textbf{Programa en Matlab/Octave:}
\begin{matlabcode}
la=(4*1/65)/(0.5^2);A=[];n=4;h=(2-0)/n;
A(1,1)=1-2*la;A(1,2)=la;
for i=2:n-2
    A(i,i-1)=la;A(i,i)=1-2*la;A(i,i+1)=la;
end
A(n-1,n-2)=la;A(n-1,n-1)=1-2*la;
x=0:h:2;
w0=x.^2-2*x;w0=w0';B=[];w0=w0(2:n);B(:,1)=w0;
for i=1:10
    w1=A*w0
    B(:,i+1)=w1;
    w0=w1;
end
mesh(B)
\end{matlabcode}

\textbf{Manual:}

Para ejecutar el programa, siga estos pasos:

\begin{enumerate}
  \item Abra Matlab o Octave en su computadora.
  \item Copie el código proporcionado en una nueva ventana de script y guárdelo como \texttt{dif\_finitas.m}.
  \item Ejecute el script presionando el botón de ejecución o escribiendo \texttt{dif\_finitas} en la consola.
  \item Observe la salida numérica en la consola, que muestra la evolución de la solución en cada paso temporal.
  \item Se generará una gráfica de tipo \texttt{mesh} que representa la evolución de la solución en la malla discreta.
\end{enumerate}

Asegúrese de tener instalado Matlab o GNU Octave y de que el archivo se encuentre en el directorio de trabajo.
\textbf{Corrida del Programa:}
\begin{matlaboutput}
w1 =

  -0.6269
  -0.8769
  -0.6269

w1 =

  -0.5341
  -0.7538
  -0.5341

w1 =

  -0.4567
  -0.6457
  -0.4567

w1 =

  -0.3908
  -0.5527
  -0.3908

w1 =

  -0.3345
  -0.4730
  -0.3345

w1 =

  -0.2862
  -0.4048
  -0.2862

w1 =

  -0.2450
  -0.3464
  -0.2450

w1 =

  -0.2096
  -0.2965
  -0.2096

w1 =

  -0.1794
  -0.2537
  -0.1794

w1 =

  -0.1535
  -0.2171
  -0.1535
\end{matlaboutput}
\begin{figure}[ht]
    \centering
    \includegraphics[width=0.8\textwidth]{im5.pdf}
    \caption{Solución de la EDP}
    \label{fig:solucion_edp5}
\end{figure}

\newpage
\section{Colocación Base}
\subsection{Programa 1}
Este informe presenta la resolución de una ecuación diferencial ordinaria (EDO) de segundo orden utilizando el método de colocación de base. Se emplean funciones base y sus derivadas para aproximar la solución, implementando el procedimiento en Matlab/Octave.

\textbf{Variables:} \\
\begin{itemize}
    \item \texttt{u1, u2}: Funciones base utilizadas para aproximar la solución.
    \item \texttt{du1, du2}: Derivadas de las funciones base.
    \item \texttt{ddu1, ddu2}: Segundas derivadas de las funciones base.
    \item \texttt{p, q}: Coeficientes de la EDO.
    \item \texttt{A}: Matriz de coeficientes del sistema lineal.
    \item \texttt{b}: Vector de términos independientes.
    \item \texttt{x}: Puntos de colocación y vector de evaluación.
    \item \texttt{c}: Coeficientes de la combinación lineal de las bases.
    \item \texttt{y}: Aproximación de la solución de la EDO.
\end{itemize}

\textbf{Programa en Matlab/Octave:}
\begin{matlabcode}
%19/06/2025
%colocacion base
%bases
u1=@(x) (1-x).*x;
u2=@(x) u1(x).*x;
%derivadas
du1=@(x) (1-2*x);
du2=@(x) 2*x-3*x.*x;
%segundas derivadas
ddu1=@(x) -2;
ddu2=@(x) 2-6*x;
%y''+y'-2y=x y(0)=0=y(1)
p=1;q=-2;
A=[];b=[];x=[1/2 4/3];
A(1,1)=ddu1(x(1))+p*du1(x(1))+q*u1(x(1));
A(1,2)=ddu2(x(1))+p*du2(x(1))+q*u2(x(1));
A(2,1)=ddu1(x(2))+p*du1(x(2))+q*u1(x(2));
A(2,2)=ddu2(x(2))+p*du2(x(2))+q*u2(x(2));
b(1)=x(1);b(2)=x(2);
c=inv(A)*b';
x=0:0.01:1;
y=c(1)*u1(x)+c(2)*u2(x);
plot(x,y)
hold on
%tarea, hacer que funciones para {1/4 1/2 3/4}
%u3=(1-x)x^3
\end{matlabcode}

\textbf{Manual:}

Para ejecutar el programa en Matlab o Octave, siga estos pasos:

\begin{enumerate}
    \item Copie el código proporcionado en una nueva ventana de script.
    \item Guarde el archivo con extensión \texttt{.m}, por ejemplo, \texttt{colocacion\_base.m}.
    \item Ejecute el script en el entorno Matlab/Octave. El programa calculará la aproximación de la solución de la EDO y mostrará la gráfica correspondiente.
    \item Puede modificar los puntos de colocación en el vector \texttt{x} y agregar nuevas funciones base para experimentar con diferentes aproximaciones.
\end{enumerate}

El gráfico generado muestra la solución aproximada obtenida mediante el método de colocación de base.
\textbf{Corrida del Programa:}
\begin{figure}[ht]
    \centering
    \includegraphics[width=0.8\textwidth]{im1.pdf}
    \caption{Solución de la EDO}
    \label{fig:solucion_edo}
\end{figure}

\subsection{Programa 2}
Este informe presenta la resolución de una ecuación diferencial ordinaria (EDO) utilizando el método de colocación con funciones base polinomiales. Se desarrolla el procedimiento teórico y se implementa el algoritmo en Matlab/Octave para obtener la solución aproximada.

\textbf{Variables:}

\begin{itemize}
    \item $u_1(x), u_2(x)$: Funciones base utilizadas en el método de colocación.
    \item $du_1(x), du_2(x)$: Derivadas de las funciones base.
    \item $ddu_1(x), ddu_2(x)$: Segundas derivadas de las funciones base.
    \item $A$: Matriz de coeficientes del sistema lineal generado por el método.
    \item $b$: Vector de términos independientes.
    \item $p, q$: Coeficientes de la EDO.
    \item $c$: Vector de coeficientes de la combinación lineal de las funciones base.
    \item $x$: Vector de puntos en el intervalo de solución.
    \item $y$: Solución aproximada de la EDO en los puntos $x$.
\end{itemize}

\textbf{Programa en Matlab/Octave:}
\begin{matlabcode}
%19/06/2025
u1=@(x) (1-x).*x;u2=@(x) u1(x).*x;
du1=@(x) (1-2.*x);du2=@(x) 2.*x-3.*x.*x;
ddu1=@(x) -2;ddu2=@(x) 2-6.*x;
A=[]; b=[]; p=1; q=-2;
pp=@(x) (ddu1(x)+p.*du1(x)+q.*u1(x)).*u1(x);
A(1,1)=quad(pp,0,1);
pp=@(x) (ddu2(x)+p.*du2(x)+q.*u2(x)).*u1(x);
A(1,2)=quad(pp,0,1);
pp=@(x) (ddu1(x)+p.*du1(x)+q.*u1(x)).*u2(x);
A(2,1)=quad(pp,0,1);
pp=@(x) (ddu2(x)+p.*du2(x)+q.*u2(x)).*u2(x);
A(2,2)=quad(pp,0,1);
pp=@(x) x.*u1(x);
b(1)=quad(pp,0,1);
pp=@(x) x.*u2(x);
b(2)=quad(pp,0,1);
c=A\b';
x=0:0.001:1;
y=c(1).*u1(x)+c(2).*u2(x);
plot(x,y,'m');
grid;
\end{matlabcode}

\textbf{Manual:}

\begin{enumerate}
    \item Definir las funciones base $u_1(x)$ y $u_2(x)$, junto con sus derivadas primeras y segundas.
    \item Plantear la ecuación diferencial ordinaria y los coeficientes $p$ y $q$.
    \item Construir la matriz de coeficientes $A$ y el vector de términos independientes $b$ mediante la integración de los productos adecuados.
    \item Resolver el sistema lineal $A c = b$ para obtener los coeficientes $c$ de la solución aproximada.
    \item Evaluar la solución aproximada $y(x)$ en el intervalo deseado utilizando la combinación lineal de las funciones base.
    \item Graficar la solución obtenida para visualizar el comportamiento de la EDO resuelta.
\end{enumerate}
\textbf{Corrida del Programa:}

\begin{figure}[ht]
    \centering
    \includegraphics[width=0.8\textwidth]{im2.pdf}
    \caption{Solución de la EDO}
    \label{fig:solucion_edo2}
\end{figure}

\section{Rayleigth-Ritz}
\subsection{Programa 1}
Es un \textit{script} que calcula las cotas del error del método de Euler para una ecuación diferencial ordinaria (EDO) dada. El método de Euler es un método numérico para resolver EDOs, y las cotas del error proporcionan una estimación de la precisión de la solución aproximada en comparación con la solución exacta.
\begin{theo}{}{}
Suponga que \textit{f} es continua y satisface la condición de Lipschitz con constante $L$ en 
\begin{equation*}
D=\{(t,y)|a\leq t\leq b \text{ y }-\infty<y<\infty\}
\end{equation*}
y que existe una constante $M$ con
\begin{equation*}
|y''(t)|\leq M,\text{ para todas las }t\in[a,b],
\end{equation*}
donde $y(t)$ denota la única solución para el problema de valor inicial
\begin{equation*}
y'=f(t,y),\quad a\leq t\leq b,\quad y(a)=\alpha.
\end{equation*}
Sean $w_0,w_1,...,w_N$ las aproximaciones generadas por el método de Euler para un entero positivo $N$. Entonces, para cada $i=0,1,2,\cdots,N,$
\begin{equation}
|y(t_i)-w_i|\leq\frac{hM}{2L}\left[e^{L(t_i-a)}-1\right]
\end{equation}
\end{theo}
\textbf{Ejemplo:} Encontrar las cotas del error de Euler para la ecuación diferencial
\begin{equation*}
y' = t - y, \quad y(0) = 0 \text{ con } 0 \leq t \leq 1
\end{equation*}
\textbf{Solución:}

La solución analítica de la ecuación diferencial es
\begin{equation*}
y = t - 1 + e^{-t}
\end{equation*}
y se quiere comparar con la solución aproximada.

Creamos un script llamado \textbf{cotas\_euler.m} que llama a la función \textbf{euler.m} de la sección anterior.

\begin{matlabcode}
%a,b: intervalo donde se define la EDO
%M : constante
%N : numero de pasos para h=(b-a)/N
%L : constante de Lipschitz
%f : f(x,y) de y'=f(x,y)
%y : solucion analitica del problema de valor inicial
%y0=alpha : condicion inicial
a=0; b=1; %t en [a,b]
f = @(t,y) t-y; %funcion f(t,y) de la EDO
M=1;
L=1;
N=3; %para h=1/3
y = @(t) t-1+exp(-t); %sol. analitica de la EDO
y0=0; %condicion inicial
%-----------------------------
% Cotas del error de Euler
%-----------------------------
h=(b-a)/N;
[t,w]=euler(a,b,N,y0,f);
fprintf('%8s %15s %20s\n', 't_i', '|y_i - w_i|', 'cota de error');
for i = 1:N
    ti = t(i);
    err = abs(y(ti) - w(i));
    bound = (h * M / (2 * L)) * abs(exp(L * (ti - a)) - 1);
    fprintf('%8.4f %15.6f %20.6f\n', ti, err, bound);
end
\end{matlabcode}
\textbf{Resultados:}
\begin{matlaboutput}
>> cotas_euler
     t_i     |y_i - w_i|        cota de error
  0.0000        0.000000             0.000000
  0.3333        0.049865             0.065935
  0.6667        0.068973             0.157956
\end{matlaboutput}

\subsection{Programa 2}
Este programa implementa el método del disparo para resolver un problema de valor en la frontera (boundary value problem, BVP) para un sistema de ecuaciones diferenciales de primer orden. El método convierte el BVP en un problema de valor inicial (IVP), que luego se resuelve usando el método de Runge-Kutta de orden 4.

En particular, el sistema resuelto es:
\[
\begin{cases}
x' = 2x + 3y \\
y' = \frac{2}{3}x + 3y \\
x(0) = 0, \quad y(0) = 1, \quad t \in [0,1]
\end{cases}
\]

Este tipo de técnica es común cuando se conocen condiciones en los extremos del intervalo y se requiere una solución aproximada de alta precisión.

---

\textbf{Variables:}

\begin{itemize}
    \item \texttt{a}, \texttt{b}: Extremos del intervalo de integración en el tiempo (\( t \in [a, b] \)).
    \item \texttt{x}, \texttt{y}: Condiciones iniciales \( x(a) \), \( y(a) \).
    \item \texttt{n}: Número de subintervalos en los que se divide el intervalo \([a, b]\); define la precisión del método.
    \item \texttt{h}: Tamaño del paso \( h = \frac{b - a}{n} \).
    \item \texttt{t}: Variable del tiempo que avanza en cada iteración.
    \item \texttt{x0}, \texttt{y0}: Valores actuales de las variables \( x \) e \( y \) en cada paso.
    \item \texttt{xt}, \texttt{yt}: Vectores donde se almacenan los valores aproximados de \( x(t) \) y \( y(t) \).
    \item \texttt{kij}: Coeficientes intermedios del método de Runge-Kutta para calcular la solución numérica, tanto para \( x \) como para \( y \).
    \item \texttt{fff(t,x,y)}: Función que representa la derivada de \( x \) en el sistema (es decir, \( x' = f(x,y) \)).
    \item \texttt{fffg(t,x,y)}: Función que representa la derivada de \( y \) en el sistema (es decir, \( y' = g(x,y) \)).
    \item \texttt{yy}: Matriz de dos filas: la primera contiene los valores de \( x(t) \), y la segunda los de \( y(t) \).
\end{itemize}

---


\textbf{Programa en Matlab/Octave:}
\begin{matlabcode}
% x' = 2x   + 3y
% y' = 2x/3 + 3y ;       x(0)=0;  y(0)=1;  t en [0,1]
function yy=disparo_r_k_11(a,b,x,y,n)
h=(b-a)/n; t=a;
x0=x; y0=y; xt(1)=x0; yt(1)=y0;
% k11 k12 k13 k14 para fff(t)
% k21  k22  k23 k24 para fffg(t)
for i=1:n
  k11 = h*fff(t, x0,y0);                  k21=h*fffg(t,x0,y0);
  k12 = h*fff(t+h/2, x0+k11/2,y0+k21/2);  k22=h*fffg(t+h/2,x0+k11/2,y0+k21/2);
  k13 = h*fff(t+h/2, x0+k12/2,y0+k22/2);  k23=h*fffg(t+h/2,x0+k12/2,y0+k22/2);
  k14 = h*fff(t+h, x0+k13,y0+k23);        k24=h*fffg(t+h,x0+k13,y0+k23);
  xs=x0+(k11 + 2*(k12+k13) + k14)/6; xt(i+1)=xs;
  ys=y0+(k21 + 2*(k22+k23) + k24)/6; yt(i+1)=ys;
  t=t+h; x0=xs; y0=ys;
end;
yy=[xt ; yt];

\end{matlabcode}

\textbf{Manual:}
Se escribe un nuevo script de nombre disparo\_r\_k\_11.m, que es el programa que resuelve el problema de valor de frontera con el método de Runge-Kutta de orden 4. Este programa usa las funciones fff(t,x,y) y fffg(t,x,y) que se escribieron en los informes siguientes.

\textbf{Corrida del Programa:}
Este programa no se ejecuta directamente, sino que se llama desde otro programa de nombre llamador.m.

\section{Crank-Nicolson para una EDP en especifico}
Este conjunto de programas resuelve un problema de valor en la frontera no lineal (BVP) de la forma:

\[
y'' + 3y' + 2y = t, \quad y(a)=ya, \quad y(b)=yb,
\]

utilizando el método del disparo, que transforma el problema BVP en un problema de valor inicial (IVP).  
La idea es proponer una pendiente inicial \( y'(a) = s \), resolver el sistema con esta condición inicial mediante el método de Runge-Kutta, y ajustar \( s \) iterativamente hasta que la solución cumpla con la condición de frontera en \( b \).

Se implementa una versión no lineal del método del disparo, en la cual se aplica el método de la secante para aproximar la pendiente correcta que satisface la condición final. La no linealidad del problema requiere un ajuste iterativo de esta pendiente hasta obtener el valor deseado en \( y(b) \).

---

\textbf{Variables:}

\begin{itemize}
    \item \texttt{a}, \texttt{b}: Extremos del intervalo del dominio de la solución.
    \item \texttt{n}: Número de subintervalos para la discretización del intervalo.
    \item \texttt{h}: Tamaño del paso, calculado como \( h = (b-a)/n \).
    \item \texttt{y1}, \texttt{y2}: Pendientes iniciales propuestas para el disparo, que se actualizan con el método de la secante.
    \item \texttt{k}: Parámetro que representa el control de la no homogeneidad del sistema (presencia del término \( r(t) \)).
    \item \texttt{runge\_sis\_}: Función externa que aplica el método de Runge-Kutta de cuarto orden para resolver el sistema asociado a una EDO.
    \item \texttt{y11}, \texttt{y22}: Soluciones del IVP con distintas pendientes iniciales.
    \item \texttt{ys1}: Nueva pendiente calculada mediante el método de la secante.
    \item \texttt{ff}: Función que define la EDO como \( y'' = f(t, y, y') \), y que se implementa por separado en \texttt{ff.m}.
\end{itemize}

\textbf{Programa en Matlab/Octave de la función ff.m:}
\begin{matlabcode}
% AQUI PONGA LA FUNCION DE y' = ay + r(t)
% por ejemplo:   y'' +3y' + 2y = t
% cambio de variable   y1 = y',  y1' = -2*y -3*y1 + t * k
%   x1=@(t,x,y)    y;
%   y1=@(t,x,y) -2*y - 3* y1 + t * k;
%
function yy=ff(t, y , y1 , k)  % k para r(t) <> 0   o  r(t)=0
   yy =  -2*y - 3.*y1 + t;
end
\end{matlabcode}
\textbf{Programa en Matlab/Octave de Disparo\_No\_lineal.m:}
\begin{matlabcode}
% METODO DEL DISPARO NO LINEAL
% por ejemplo:   y'' +0y' + y^2/100 = t;  y(a)=ya ,  y(b)=yb;
% cambio de variable   y1 = y',  y1' = -2*y -3*y1 + t * k
%   x1=@(t,x,y)    y;
%   y1=@(t,x,y) x*x/10  + t ;
%
% % llama a funtion yy=runge_sis_(a,b,x,y,n,k)
%
%  y(t)  =  y1(t) + (yb - y1(b))*y2(t)/y2(b) ,  es la solucion
%
clear all;
a=0; b=1; n=8;  h=(b-a)/n;  err1=.00000001;
x=0; y=01/2;    B=y;% fronteras
% ingresar pendiente 1  y 2
y1=.3; y2=4; %y11=y22=[ ];
k=1;  %   pendiente yn
d=runge_sis_(a,b,x,y1,n,k);
y11=d(1,:);

for i=1: 8000


   % para y2(t)   no homogeneo  r(t) <> 0
   % pendiente yn1
   d=runge_sis_(a,b,x,y2,n,k);
   y22=d(1,:);
   % SOLUCION
   ys1 = y2 - (y22(n+1) - B)/(y22(n+1)-y11(n+1))*(y2-y1);
   printf('%d  yn=%f  yn+1 =%f y(b)=%f \n',i , y2, ys1, y22(n+1));
   if abs(ys1-y2)<err1 break; end;
   y1=y2; y2=ys1; y11=y22;
end; ys1, d(1, n+1)
i, y22
t=a:h:b;
plot(t,y22);
grid;
\end{matlabcode}

\textbf{Manual:}

1. Guarde los archivos como \texttt{ff.m} y \texttt{Disparo\_No\_lineal.m}.

2. Asegúrese de tener la función \texttt{runge\_sis\_} correctamente implementada y disponible en el mismo directorio.

3. Ejecute el programa principal con:
\begin{verbatim}
>> Disparo_No_lineal
\end{verbatim}

4. El script ajusta la pendiente inicial para que la solución del problema de valor inicial cumpla con la condición de frontera en \( t = b \).

5. Al finalizar, se grafica la solución aproximada.

\textbf{Corrida del Programa:}
\begin{matlaboutput}
1  yn=4.000000  yn+1 =1.788797 y(b)=1.014180 
2  yn=1.788797  yn+1 =1.788797 y(b)=0.500000 
ys1 = 1.7888
ans = 0.5000
i = 2
y22 =

        0   0.1858   0.3103   0.3911   0.4414   0.4711   0.4874   0.4957   0.5000
\end{matlaboutput}
\begin{figure}[ht]
\centering
\includegraphics[width=0.5\textwidth]{im6_2.pdf}
\end{figure}

\section{Crank-Nicolson para una EDP en general con $\theta=1/2$}
El método Crank-Nicolson es un esquema numérico implícito para resolver ecuaciones en derivadas parciales (EDP) de tipo parabolico, como la ecuación de difusión o calor. Es una combinación de los métodos explícito e implícito, usando el promedio entre ambos en cada paso temporal. Para $\theta=1/2$, se obtiene el método clásico de Crank-Nicolson, que es estable y de segundo orden en el tiempo y espacio.

\textbf{Variables:}

\begin{itemize}
    \item $a$, $b$: Extremos del intervalo espacial.
    \item $n$: Número de subintervalos espaciales.
    \item $h$: Tamaño de paso espacial, $h=(b-a)/n$.
    \item $tt$: Tiempo final de la simulación.
    \item $k$: Tamaño de paso temporal.
    \item $\alpha$: Parámetro de difusión ($\alpha^2$).
    \item $L$: Número de Fourier, $L=\alpha k/h^2$.
    \item $x$: Vector de nodos espaciales.
    \item $w0$: Condición inicial en los nodos.
    \item $A$, $B$: Matrices del sistema lineal para el método.
    \item $BB$: Matriz que almacena la solución en cada paso temporal.
    \item $m$: Número de pasos temporales.
\end{itemize}

\textbf{Programa en Matlab/Octave:}
\begin{matlabcode}
%Crank_nicolson general para theta=1/2
a=0;b=1;n=10;h=(b-a)/n;
tt=1;k=1/12; alp=1; %alp: alpha cuadrado
L=alp*k/h^2;BB=A=B=[];bb=[];
x=a:h:b;w0=x-x.^2; BB(:,1)=w0;
w0=w0(2:n)';
A(1,1)=1+L;A(1,2)=-L/2;
B(1,1)=1-L;B(1,2)=L/2;
for i=2:n-2
    A(i,i-1)=-L/2;A(i,i)=1+L;A(i,i+1)=-L/2;
    B(i,i-1)=L/2;B(i,i)=1-L;B(i,i+1)=L/2;
end
i=n-1;
A(i,i-1)=-L/2;A(i,i)=1+L;
B(i,i-1)=L/2;B(i,i)=1-L;
m=ceil(tt/k);
for i=1:m
    wx=B*w0;wx=A\wx;BB(:,i+1)=[0 wx' 0]';
    w0=wx;
end
BB
mesh(BB)
\end{matlabcode}

\textbf{Manual:}

Para ejecutar el programa, siga estos pasos:

1. Copie el código Matlab/Octave en un archivo llamado `crank\_nicolson.m`.
2. Abra Matlab o GNU Octave y navegue hasta el directorio donde guardó el archivo.
3. Ejecute el archivo escribiendo `crank\_nicolson` en la consola.
4. El resultado será la matriz `BB`, que contiene la solución numérica en cada paso temporal, y una gráfica de la evolución de la solución.

La matriz `BB` muestra cómo la condición inicial evoluciona en el tiempo bajo el método Crank-Nicolson. La gráfica generada (`mesh(BB)`) permite visualizar la difusión de la solución en el dominio espacio-tiempo.
\textbf{Corrida del Programa:}
\begin{matlaboutput}
BB =

 Columns 1 through 8:

        0        0        0        0        0        0        0        0
   0.0900   0.0272   0.0181   0.0033   0.0043  -0.0002   0.0014  -0.0005
   0.1600   0.0594   0.0290   0.0102   0.0052   0.0019   0.0008   0.0005
   0.2100   0.0875   0.0368   0.0158   0.0062   0.0030   0.0009   0.0007
   0.2400   0.1061   0.0419   0.0191   0.0073   0.0034   0.0013   0.0006
   0.2500   0.1126   0.0437   0.0201   0.0077   0.0035   0.0015   0.0005
   0.2400   0.1061   0.0419   0.0191   0.0073   0.0034   0.0013   0.0006
   0.2100   0.0875   0.0368   0.0158   0.0062   0.0030   0.0009   0.0007
   0.1600   0.0594   0.0290   0.0102   0.0052   0.0019   0.0008   0.0005
   0.0900   0.0272   0.0181   0.0033   0.0043  -0.0002   0.0014  -0.0005
        0        0        0        0        0        0        0        0

 Columns 9 through 13:

        0        0        0        0        0
   0.0006  -0.0004   0.0003  -0.0003   0.0002
   0.0000   0.0002  -0.0001   0.0001  -0.0001
   0.0000   0.0002  -0.0001   0.0001  -0.0000
   0.0002   0.0001   0.0001  -0.0000   0.0000
   0.0003   0.0000   0.0001  -0.0000   0.0000
   0.0002   0.0001   0.0001  -0.0000   0.0000
   0.0000   0.0002  -0.0001   0.0001  -0.0000
   0.0000   0.0002  -0.0001   0.0001  -0.0001
   0.0006  -0.0004   0.0003  -0.0003   0.0002
        0        0        0        0        0
\end{matlaboutput}
\begin{figure}[ht]
    \centering
    \includegraphics[width=0.8\textwidth]{im12.pdf}
    \caption{Solución de la EDP con Crank-Nicolson}
    \label{fig:solucion_edp12}
\end{figure}


\end{document}
