Este programa implementa el método de Runge-Kutta de cuarto orden para resolver un sistema de dos ecuaciones diferenciales acopladas. Se utiliza para aproximar soluciones numéricas de sistemas de la forma:
\[
\begin{cases}
x' = f_1(t,x,y) \\
y' = f_2(t,x,y)
\end{cases}
\]
Sin embargo, en esta implementación, se asume que la EDO a resolver está definida mediante una función auxiliar llamada \texttt{ff.m}, y se usa el mismo valor de derivada para ambas componentes (lo cual puede corresponder a un modelo simplificado o una EDO escalar replicada).

\textbf{Variables:}

\begin{itemize}
    \item \texttt{a}, \texttt{b}: Extremos del intervalo en el cual se resuelve la EDO.
    \item \texttt{x0}, \texttt{y0}: Condiciones iniciales de las variables dependientes \(x\) e \(y\) en \(t = a\).
    \item \texttt{n}: Número de pasos que se utilizarán en el método (mayor \(n\) → mayor precisión).
    \item \texttt{k}: Parámetro adicional que se pasa a la función \texttt{ff} (usualmente un parámetro del modelo).
    \item \texttt{h}: Tamaño del paso, calculado como \(h = \frac{b - a}{n}\).
    \item \texttt{xt}, \texttt{yt}: Vectores que almacenan los valores aproximados de \(x(t)\) y \(y(t)\).
    \item \texttt{kkij}: Coeficientes intermedios del método de Runge-Kutta (según el índice correspondiente).
    \item \texttt{xs}, \texttt{ys}: Aproximaciones actualizadas de \(x\) e \(y\) para el siguiente paso.
    \item \texttt{yy}: Matriz que contiene los resultados finales: la primera fila corresponde a los valores de \(x\), y la segunda fila a los valores de \(y\).
\end{itemize}

\textbf{Programa en Matlab/Octave:}
\begin{matlabcode}
function yy=runge_sis(a,b,x0,y0,n,k)	
%ff es la ED a resolver definida en el archivo ff.m
h=(b-a)/n;
xt=[];yt=[];t=a;
xt(1)=x0;yt(1)=y0;
%k11 k21 k31 k41 para la primera
%k12 k22 k32 k42 para la segunda
for i=1:n
    k11=h*x0;
    k12=h*ff(t,x0,y0,k);
    k21=h*ff(t+h/2,x0+k11/2,y0+k12/2,k);
    k22=h*ff(t+h/2,x0+k11/2,y0+k12/2,k);
    k31=h*ff(t+h/2,x0+k21/2,y0+k22/2,k);
    k32=h*ff(t+h/2,x0+k21/2,y0+k22/2,k);
    k41=h*ff(t+h,x0+k31,y0+k32,k);
    k42=h*ff(t+h,x0+k31,y0+k32,k);
    xs=x0+(k11+2*(k21+k31)+k41)/6;
    ys=y0+(k12+2*(k22+k32)+k42)/6;
    xt(i+1)=xs;yt(i+1)=ys;
    t=t+h;x0=xs;y0=ys;
end%i
yy=[xt;yt];

\end{matlabcode}
\textbf{Manual:}

1. Este archivo debe guardarse como \texttt{runge\_sis.m}.

2. Debe estar acompañado de un archivo separado llamado \texttt{ff.m} que defina la función de la EDO.

3. No se ejecuta directamente desde la consola, sino que se llama desde otro script o función, por ejemplo:
\begin{verbatim}
>> ff = @(t,x,y,k) ...   % definir si no se usa ff.m
>> runge_sis(0,1,1,0,100,2)
\end{verbatim}
4. El resultado es una matriz donde la primera fila contiene los valores de \(x\) y la segunda los de \(y\), evaluados en \(n+1\) puntos uniformemente distribuidos entre \(a\) y \(b\).


\textbf{Corrida del Programa:} Este programa no se ejecuta directamente, sino que se llama desde otro programa.
