Este informe presenta la resolución de una ecuación diferencial ordinaria (EDO) utilizando el método de colocación con funciones base polinomiales. Se desarrolla el procedimiento teórico y se implementa el algoritmo en Matlab/Octave para obtener la solución aproximada.

\textbf{Variables:}

\begin{itemize}
    \item $u_1(x), u_2(x)$: Funciones base utilizadas en el método de colocación.
    \item $du_1(x), du_2(x)$: Derivadas de las funciones base.
    \item $ddu_1(x), ddu_2(x)$: Segundas derivadas de las funciones base.
    \item $A$: Matriz de coeficientes del sistema lineal generado por el método.
    \item $b$: Vector de términos independientes.
    \item $p, q$: Coeficientes de la EDO.
    \item $c$: Vector de coeficientes de la combinación lineal de las funciones base.
    \item $x$: Vector de puntos en el intervalo de solución.
    \item $y$: Solución aproximada de la EDO en los puntos $x$.
\end{itemize}

\textbf{Programa en Matlab/Octave:}
\begin{matlabcode}
%19/06/2025
u1=@(x) (1-x).*x;u2=@(x) u1(x).*x;
du1=@(x) (1-2.*x);du2=@(x) 2.*x-3.*x.*x;
ddu1=@(x) -2;ddu2=@(x) 2-6.*x;
A=[]; b=[]; p=1; q=-2;
pp=@(x) (ddu1(x)+p.*du1(x)+q.*u1(x)).*u1(x);
A(1,1)=quad(pp,0,1);
pp=@(x) (ddu2(x)+p.*du2(x)+q.*u2(x)).*u1(x);
A(1,2)=quad(pp,0,1);
pp=@(x) (ddu1(x)+p.*du1(x)+q.*u1(x)).*u2(x);
A(2,1)=quad(pp,0,1);
pp=@(x) (ddu2(x)+p.*du2(x)+q.*u2(x)).*u2(x);
A(2,2)=quad(pp,0,1);
pp=@(x) x.*u1(x);
b(1)=quad(pp,0,1);
pp=@(x) x.*u2(x);
b(2)=quad(pp,0,1);
c=A\b';
x=0:0.001:1;
y=c(1).*u1(x)+c(2).*u2(x);
plot(x,y,'m');
grid;
\end{matlabcode}

\textbf{Manual:}

\begin{enumerate}
    \item Definir las funciones base $u_1(x)$ y $u_2(x)$, junto con sus derivadas primeras y segundas.
    \item Plantear la ecuación diferencial ordinaria y los coeficientes $p$ y $q$.
    \item Construir la matriz de coeficientes $A$ y el vector de términos independientes $b$ mediante la integración de los productos adecuados.
    \item Resolver el sistema lineal $A c = b$ para obtener los coeficientes $c$ de la solución aproximada.
    \item Evaluar la solución aproximada $y(x)$ en el intervalo deseado utilizando la combinación lineal de las funciones base.
    \item Graficar la solución obtenida para visualizar el comportamiento de la EDO resuelta.
\end{enumerate}
\textbf{Corrida del Programa:}

\begin{figure}[ht]
    \centering
    \includegraphics[width=0.8\textwidth]{im2.pdf}
    \caption{Solución de la EDO}
    \label{fig:solucion_edo2}
\end{figure}
