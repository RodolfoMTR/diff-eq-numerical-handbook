Es un método numérico para resolver ecuaciones diferenciales ordinarias (EDO) de la forma
\begin{equation*}
    y' = f(t,y), \quad y(t_0) = y_0 \quad  a\leq t\leq b
\end{equation*}
donde $f(t,y)$ es una función continua y $y_0$ es el valor inicial de la solución en el punto $t_0$.

El método Runge-Kutta de orden 2 es una extensión del método de Euler, proporcionando una mayor precisión en la aproximación de la solución.

\noindent\textbf{RUNGE-KUTTA de orden 2}
\begin{equation}
    y_{i+1}=y_i+(a_1k_1+a_2k_2)h
\end{equation}
donde:
\begin{align*}
    k_1&=f(x_i,y_i)\\
    k_2&=f(x_i+\frac{1}{2}h,y_i+\frac{1}{2}k_1h)\\
    h&=(b-a)/N
\end{align*}
Éste es el método del punto medio.\\
Sea la EDO
\begin{equation*}
    y'=t-y^{1.5} \quad y(1)=1 \text{ con } 1 \leq t \leq 2
\end{equation*}
Hacemos el Algoritmo Runge-Kutta de orden 2

\textbf{Código en Matlab/Octave:}
\begin{matlabcode}
    %Runge Kutta de orden 2
    y1 = @(t,y) t-y^(3/2); %ec diff ordinaria
    y0=1; %condición inicial
    a=1;b=2; %intervalo donde se quiere la solución
    %-----------------------------------------
    N = 8;
    yt= zeros(1,N); yt(1)=y0;
    h = (b-a)/N;
    a1=a;
    for i=1:N
        k1=h*y1(a1,y0);
        k2=h*y1(a1+h,y0+k1);
        yp=y0+(k1+k2)/2;yt(i+1)=yp;
        fprintf('%d| yn=%f| k1=%f| k2=%f| y_n+1=%f\n',i,y0,k1,k2,yp)
        y0=yp; a1=a1+h;
    end
     t1=a:h:b;
     plot(t1,yt,'.r','MarkerSize',10)
\end{matlabcode}
\textbf{Manual:}
\begin{itemize}
    \item Abrir el archivo \textbf{Rkorden2.m} en Matlab/Octave.
    \item En las lineas 
    \begin{verbatim}
y1 = @(t,y) t-y^(3/2); %ec diff ordinaria
y0=1; %condición inicial
a=1;b=2; %intervalo donde se quiere la solución
    \end{verbatim}
    Se ingresa la ecuación diferencial ordinaria a solucionar, la condición inicial y el intervalo de solución.
    \item En la línea \textbf{N=8;} se define el número de pasos a realizar.
    \item Ejecutar el script con el comando \textbf{Rkorden2} en la consola de Matlab/Octave.
\end{itemize}
\textbf{Corrida del Programa:}
\begin{matlaboutput}
    octave:12> RKorden2
    1| yn=1.000000| k1=0.000000| k2=0.015625| y_n+1=1.007812
    2| yn=1.007812| k1=0.014157| k2=0.027108| y_n+1=1.028445
    3| yn=1.028445| k1=0.025879| k2=0.036552| y_n+1=1.059661
    4| yn=1.059661| k1=0.035523| k2=0.044235| y_n+1=1.099540
    5| yn=1.099540| k1=0.043379| k2=0.050392| y_n+1=1.146425
    6| yn=1.146425| k1=0.049688| k2=0.055231| y_n+1=1.198885
    7| yn=1.198885| k1=0.054662| k2=0.058938| y_n+1=1.255685
    8| yn=1.255685| k1=0.058489| k2=0.061683| y_n+1=1.315771
\end{matlaboutput}
\begin{figure}[ht]
    \centering
    \includegraphics[width = 10cm]{figure1_inf5.pdf}
\end{figure}
