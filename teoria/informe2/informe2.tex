Este programa define la función \texttt{ff}, que representa una ecuación diferencial ordinaria (EDO) de segundo orden reducida a un sistema de primer orden. Esta función es utilizada por otro programa (como \texttt{runge\_sis.m}) que implementa un método numérico para resolver sistemas de EDO, como Runge-Kutta. La ecuación definida modela una relación entre la variable dependiente \( y(t) \), su derivada \( y'(t) \), el tiempo \( t \), y un parámetro \( k \). Es un modelo lineal típico en problemas de dinámica o control.

\textbf{Variables:}

\begin{itemize}
    \item \texttt{t}: Variable independiente (tiempo).
    \item \texttt{y}: Variable dependiente, que representa la función \( y(t) \).
    \item \texttt{y1}: Derivada de \( y \) con respecto a \( t \), es decir, \( y'(t) \).
    \item \texttt{k}: Parámetro que puede controlar la intensidad o el comportamiento del término externo \( t \cdot k \) en la ecuación.
    \item \texttt{yy}: Resultado de la evaluación de la ecuación diferencial en el punto \((t, y, y')\), es decir, el valor de \( y''(t) \).
\end{itemize}

\textbf{Programa en Matlab/Octave:}
\begin{matlabcode}
function yy=ff(t,y,y1,k)
        yy=-2*y-3*y1+t*k;
end
\end{matlabcode}
\textbf{Manual:}

1. Este archivo debe guardarse con el nombre \texttt{ff.m}.

2. Define la función que representa la ecuación diferencial a resolver.

3. No se ejecuta de forma autónoma desde la consola, sino que debe ser llamada desde otro programa (como \texttt{runge\_sis.m}) que implemente el método numérico para resolver sistemas de EDOs.

4. El valor retornado, \texttt{yy}, representa la derivada de segundo orden \( y''(t) \) para la ecuación:
   \[
   y''(t) = -2y(t) - 3y'(t) + tk
   \]

\textbf{Corrida del Programa:} Este programa no se ejecuta directamente, sino que se llama desde otro programa de nombre runge\_sis.m.
