El método Crank-Nicolson es un esquema numérico implícito utilizado para resolver ecuaciones diferenciales parciales, especialmente la ecuación de difusión o calor. Es conocido por ser estable y tener buena precisión, ya que combina los métodos de Euler hacia adelante y hacia atrás.

\textbf{Variables:}

\begin{itemize}
    \item $L$: Parámetro relacionado con el coeficiente de difusión y el tamaño de la malla.
    \item $A$: Matriz de coeficientes del sistema lineal generado por el método Crank-Nicolson.
    \item $CI$: Vector de condiciones iniciales.
    \item $w0$: Vector de valores de la solución en el paso anterior.
    \item $w1$: Vector de valores de la solución en el paso actual.
    \item $BB$: Matriz que almacena la evolución de la solución en cada paso de tiempo.
    \item $cc$: Vector auxiliar para guardar los resultados en cada iteración.
\end{itemize}

\textbf{Programa en Matlab/Octave:}
\begin{matlabcode}
% Usando Crank -Nicolson 10/07/2025
L = 0.02086;
A = [1-2*L, L, 0, 0;
     L, 1-2*L, L, 0;
     0, L, 1-2*L, L;
     0, 0, L, 1-2*L];

CI = [100*L; 0; 0; 50*L];
w0 = [0; 0; 0; 0];
BB(:,1) = [100; 0; 0; 0; 0; 50];

for i = 1:10
	w1 = A * w0 + CI;
	disp([100, w1', 50]);
	cc = [100, w1', 50]; % Asignar cc para guardarlo
	BB(:,i+1) = cc';
	w0 = w1;
end

surf(BB)
\end{matlabcode}

\textbf{Manual:}

El programa implementa el método de Crank-Nicolson para resolver un problema de difusión en una malla de cuatro nodos. Los pasos para ejecutar el programa son los siguientes:

\begin{enumerate}
    \item Definir el parámetro $L$ según el coeficiente de difusión y el tamaño de la malla.
    \item Construir la matriz $A$ que representa el sistema lineal generado por el método.
    \item Establecer el vector de condiciones iniciales $CI$ y el vector inicial $w0$.
    \item Inicializar la matriz $BB$ para almacenar la evolución temporal de la solución.
    \item Ejecutar el ciclo \texttt{for} para calcular la solución en cada paso de tiempo, actualizando los vectores y almacenando los resultados.
    \item Visualizar la evolución de la solución usando la función \texttt{surf}.
\end{enumerate}

Para modificar el problema, se pueden cambiar los valores de $L$, las condiciones iniciales, el número de nodos o el número de pasos de tiempo. El resultado se muestra tanto en la consola como en una gráfica tridimensional.
\textbf{Corrida del Programa:}
\begin{matlaboutput}
          100        2.086            0            0        1.043           50
          100        4.085     0.043514     0.021757       2.0425           50
          100       6.0015      0.12736     0.064363       3.0007           50
          100       7.8397      0.24858      0.12693       3.9199           50
          100       9.6038       0.4044      0.20859        4.802           50
          100       11.298      0.59221      0.30849        5.649           50
          100       12.925      0.80961      0.42581       6.4628           50
          100       14.488       1.0543      0.55975        7.245           50
          100       15.992       1.3242      0.70952       7.9974           50
          100       17.438       1.6174      0.87437       8.7216           50
\end{matlaboutput}
\begin{figure}[ht]
    \centering
    \includegraphics[width=0.8\textwidth]{media/im13.pdf}
    \caption{Gráfica de la corrida del programa}
    \label{fig:informe13}
\end{figure}
