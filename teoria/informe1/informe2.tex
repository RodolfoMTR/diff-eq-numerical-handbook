Es un método numérico para resolver una edo de la forma
\[y' = f(t,y), \quad y(t_0) = y_0\quad  a\leq t\leq b\]
esta dado en términos matemáticos
\begin{equation}
y_{i+1}=y_i+h\cdot f(x_i,y_i)
\end{equation}
donde
\begin{itemize}
    \item $y_i$ es el valor de la función en el punto $x_i$.
    \item $h$ es el tamaño del paso, que se define como $h=(b-a)/N$, donde $N$ es el número de pasos.
    \item $f(x_i,y_i)$ es la función que define la ecuación diferencial evaluada en el punto $(x_i,y_i)$.
    \item $x_i$ es el valor de la variable independiente en el paso $i$, que se define como $x_i = a + i \cdot h$.
    \item $y_{i+1}$ es el valor de la función en el siguiente paso, que se calcula a partir del valor actual $y_i$ y la pendiente dada por $f(x_i,y_i)$.
    \item $N$ es el número de pasos que se desea realizar en el intervalo $[a,b]$.
    \item $a$ es el límite inferior del intervalo.
    \item $b$ es el límite superior del intervalo.
    \item $t_0$ es el valor inicial de la variable independiente.
    \item $y_0$ es el valor inicial de la función en el punto $t_0$.
\end{itemize}

Esta formula se conoce como \textbf{método de Euler}(o de \textit{Euler-Cauchy} o de \textit{Punto pendiente})\\

\vspace{2mm}
\textbf{Ejemplo:} Resolver la EDO con el método
\begin{equation}
y' = y^{1.5} + 1, \quad y(0) = 10 \text{ con } 0 \leq t \leq 1
\end{equation}

Con $h=1/2,\, h=1/4,\, h=0.1$

\textbf{Solución}

Creamos un script en \textit{Matlab/Octave} para resolver la ecuación diferencial con nombre \textbf{euler.m}.

\textbf{Código en Matlab/Octave:}
\begin{matlabcode}
function [t,w] = euler(a,b,N,alpha,f)
	h=(b-a)/N;
	t = zeros(N+1,1); % Para guardar los valores de t
	w = zeros(N+1,1); % Para guardar los valores de w
	t(1)=a;
	w(1)=alpha;
	for i=1:N
		w(i+1)=w(i)+h*f(t(i),w(i));
		t(i+1)=a+i*h;
	end
end
\end{matlabcode}
Luego para ejecutar el script, creamos otro script llamado \textbf{run\_euler.m} que llama a la función \textbf{euler.m} y grafica los resultados.
Como se quiere para un $h=1/2$, entonces $N=2$, esto se obtiene resolviendo la formula $h=(b-a)/N$.
\begin{matlabcode}
f = @(t, y) y^(3/2) + 1; %función de la EDO
a=0;b=1; %intervalo
N=2; %número de pasos
y0=10; %condición inicial
%-----------------------------
[T1, W1] = euler(a, b, N, y0, f)
plot(T1, W1, 'ro-');
grid on;
\end{matlabcode}
\textbf{Manual:}
\begin{itemize}
    \item Abrir el archivo \textbf{run\_euler.m} en Matlab/Octave.
    \item En las lineas 
    \begin{verbatim}
f = @(t, y) y^(3/2) + 1; %función de la EDO
a=0;b=1; %intervalo
N=2; %número de pasos
y0=10; %condición inicial
    \end{verbatim}
    Se ingresa la ecuación diferencial ordinaria a solucionar, la condición inicial y el intervalo de solución.
    \item En la línea \textbf{N=2;} se define el número de pasos a realizar.
    \item Ejecutar el script con el comando \textbf{run\_euler} en la consola de Matlab/Octave.
\end{itemize}
\textbf{Corrida del Programa:}
\begin{matlaboutput}
T1 =

    0
0.5000
1.0000

W1 =

10.000
26.311
94.293
\end{matlaboutput}
\begin{figure}[ht]
\centering
\includegraphics[width=10cm]{figure1_inf3.pdf}
\caption{Gráfica de la solución con $h=1/2$}
\end{figure}
\newpage
Para $h=1/4$, entonces $N=4$.
\begin{matlabcode}
f = @(t, y) y^(3/2) + 1; %función de la EDO
a=0;b=1; %intervalo
N=4; %número de pasos
y0=10; %condición inicial
%-----------------------------
[T2, W2] = euler(a, b, N, y0, f)
plot(T2, W2, 'ro-');
grid on;
\end{matlabcode}
\textbf{Corrida del Programa:}
\begin{matlaboutput}
T2 =

    0
0.2500
0.5000
0.7500
1.0000

W2 =

10.000
18.156
37.746
95.971
331.266
\end{matlaboutput}
\begin{figure}[ht]
\centering
\includegraphics[width=10cm]{figure2_inf3.pdf}
\caption{Gráfica de la solución con $h=1/4$}
\end{figure}
Para $h=0.1$, entonces $N=10$.
\begin{matlabcode}
f = @(t, y) y^(3/2) + 1; %función de la EDO
a=0;b=1; %intervalo
N=10; %número de pasos
y0=10; %condición inicial
%-----------------------------
[T3, W3] = euler(a, b, N, y0, f)
plot(T3, W3, 'ro-');
grid on;
\end{matlabcode}
\textbf{Corrida del Programa:}
\begin{matlaboutput}
T3 =

        0
   0.1000
   0.2000
   0.3000
   0.4000
   0.5000
   0.6000
   0.7000
   0.8000
   0.9000
   1.0000

W3 =

   1.0000e+01
   1.3262e+01
   1.8192e+01
   2.6051e+01
   3.9448e+01
   6.4325e+01
   1.1601e+02
   2.4107e+02
   6.1548e+02
   2.1425e+03
   1.2060e+04
\end{matlaboutput}
\begin{figure}[ht]
\centering
\includegraphics[width=8cm]{figure3_inf3.pdf}
\caption{Gráfica de la solución con $h=0.1$}
\end{figure}
