Este informe presenta la solución numérica de una ecuación diferencial parcial (EDP) utilizando el método de diferencias finitas. El programa implementado en Matlab/Octave resuelve la EDP en un dominio discreto, mostrando la evolución de la solución en el tiempo.

\textbf{Variables:} \\
\begin{itemize}
    \item \texttt{la}: Parámetro relacionado con la estabilidad del método, calculado como $\frac{4 \times 1/1600}{0.1^2}$.
    \item \texttt{A}: Matriz de coeficientes del esquema de diferencias finitas.
    \item \texttt{n}: Número de divisiones espaciales (puntos de la malla).
    \item \texttt{h}: Tamaño de paso espacial, calculado como $(2-0)/n$.
    \item \texttt{x}: Vector de posiciones espaciales.
    \item \texttt{w0}: Vector de condiciones iniciales, definido como $x^2 - 2x$.
    \item \texttt{B}: Matriz que almacena la evolución de la solución en el tiempo.
    \item \texttt{w1}: Vector de la solución en el siguiente paso temporal.
\end{itemize}

\textbf{Programa en Matlab/Octave:}
\begin{matlabcode}
la=(4*1/1600)/(0.1^2);A=[];n=10;h=(2-0)/n;
A(1,1)=1-2*la;A(1,2)=la;
for i=2:n-2
    A(i,i-1)=la;A(i,i)=1-2*la;A(i,i+1)=la;
end
A(n-1,n-2)=la;A(n-1,n-1)=1-2*la;
x=0:h:2;
w0=x.^2-2*x;w0=w0';B=[];w0=w0(2:n);B(:,1)=w0;
for i=1:10*10
    w1=A*w0;
    B(:,i+1)=w1;
    w0=w1;
end
mesh(B)
\end{matlabcode}

\textbf{Manual:}

Para ejecutar el programa, siga estos pasos:

\begin{enumerate}
    \item Abra Matlab o GNU Octave.
    \item Copie el código proporcionado en un archivo llamado \texttt{dif\_finitas.m}.
    \item Ejecute el archivo escribiendo \texttt{dif\_finitas} en la consola.
    \item El programa mostrará una gráfica 3D (\texttt{mesh}) que representa la evolución de la solución en el tiempo.
\end{enumerate}

Asegúrese de tener los permisos necesarios para guardar archivos y visualizar gráficos en su entorno de trabajo.
\textbf{Corrida del Programa:}
\begin{figure}[ht]
    \centering
    \includegraphics[width=0.8\textwidth]{im7.pdf}
    \caption{Solución de la EDP}
    \label{fig:solucion_edp7}
\end{figure}
