Este programa implementa el método de diferencias finitas para resolver una ecuación diferencial parcial (EDP) de tipo parabólico, simulando la evolución temporal de una función en un dominio discreto. Utiliza una malla espacial y temporal para aproximar la solución numérica.

\textbf{Variables:} \\
\begin{itemize}
  \item \texttt{la}: Parámetro de estabilidad, relacionado con el paso temporal y espacial.
  \item \texttt{A}: Matriz de coeficientes del esquema de diferencias finitas.
  \item \texttt{n}: Número de intervalos espaciales.
  \item \texttt{h}: Tamaño del paso espacial.
  \item \texttt{x}: Vector de puntos espaciales.
  \item \texttt{w0}: Vector de valores iniciales de la función.
  \item \texttt{B}: Matriz que almacena la evolución de la solución en el tiempo.
  \item \texttt{w1}: Vector de la solución en el siguiente paso temporal.
\end{itemize}

\textbf{¿Qué hace?:} \\
Construye la matriz de diferencias finitas, inicializa la condición inicial, y realiza iteraciones temporales para calcular la evolución de la solución. Finalmente, muestra los resultados numéricos y grafica la solución aproximada de la EDP.

\textbf{Programa en Matlab/Octave:}
\begin{matlabcode}
la=(4*1/1600)/(0.1^2);A=[];n=10;h=(2-0)/n;
A(1,1)=1-2*la;A(1,2)=la;
for i=2:n-2
    A(i,i-1)=la;A(i,i)=1-2*la;A(i,i+1)=la;
end
A(n-1,n-2)=la;A(n-1,n-1)=1-2*la;
x=0:h:2;
w0=x.^2-2*x;w0=w0';B=[];w0=w0(2:n);B(:,1)=w0;
for i=1:40
    w1=A*w0
    B(:,i+1)=w1;
    w0=w1;
end
mesh(B)
\end{matlabcode}

\textbf{Manual:}

\textbf{1. Ingreso de datos:} \\
El usuario debe definir los parámetros del problema, como el número de intervalos espaciales (\texttt{n}), el tamaño del paso espacial (\texttt{h}), y la condición inicial (\texttt{w0}). Estos valores pueden modificarse directamente en el código.

\textbf{2. Ejecución:} \\
Ejecute el programa en Matlab u Octave. El código construye la matriz de diferencias finitas, inicializa la condición inicial y realiza las iteraciones temporales para calcular la evolución de la solución.

\textbf{3. Visualización:} \\
Al finalizar, el programa muestra los resultados numéricos de cada paso temporal y grafica la solución aproximada usando la función \texttt{mesh}, permitiendo observar la evolución de la EDP en el tiempo.

\textbf{4. Interpretación:} \\
Analice la gráfica y los valores obtenidos para comprender el comportamiento de la solución en el dominio discreto. Puede modificar los parámetros para observar diferentes escenarios y verificar la estabilidad del método.
\textbf{Corrida del Programa:}
\begin{multicols*}{3}
\begin{matlaboutput}
w1 =

  -0.3400
  -0.6200
  -0.8200
  -0.9400
  -0.9800
  -0.9400
  -0.8200
  -0.6200
  -0.3400

w1 =

  -0.3250
  -0.6000
  -0.8000
  -0.9200
  -0.9600
  -0.9200
  -0.8000
  -0.6000
  -0.3250

w1 =

  -0.3125
  -0.5813
  -0.7800
  -0.9000
  -0.9400
  -0.9000
  -0.7800
  -0.5812
  -0.3125

w1 =

  -0.3016
  -0.5637
  -0.7603
  -0.8800
  -0.9200
  -0.8800
  -0.7603
  -0.5637
  -0.3016

w1 =

  -0.2917
  -0.5473
  -0.7411
  -0.8601
  -0.9000
  -0.8601
  -0.7411
  -0.5473
  -0.2917

w1 =

  -0.2827
  -0.5319
  -0.7224
  -0.8403
  -0.8800
  -0.8403
  -0.7224
  -0.5319
  -0.2827

w1 =

  -0.2743
  -0.5172
  -0.7042
  -0.8208
  -0.8602
  -0.8208
  -0.7042
  -0.5172
  -0.2743

w1 =

  -0.2665
  -0.5032
  -0.6866
  -0.8015
  -0.8405
  -0.8015
  -0.6866
  -0.5032
  -0.2665

w1 =

  -0.2590
  -0.4899
  -0.6695
  -0.7825
  -0.8210
  -0.7825
  -0.6695
  -0.4899
  -0.2590

w1 =

  -0.2520
  -0.4771
  -0.6528
  -0.7639
  -0.8017
  -0.7639
  -0.6528
  -0.4771
  -0.2520

w1 =

  -0.2453
  -0.4648
  -0.6367
  -0.7456
  -0.7828
  -0.7456
  -0.6367
  -0.4648
  -0.2453

w1 =

  -0.2388
  -0.4529
  -0.6209
  -0.7277
  -0.7642
  -0.7277
  -0.6209
  -0.4529
  -0.2388

w1 =

  -0.2326
  -0.4414
  -0.6056
  -0.7101
  -0.7459
  -0.7101
  -0.6056
  -0.4414
  -0.2326

w1 =

  -0.2267
  -0.4302
  -0.5907
  -0.6929
  -0.7280
  -0.6929
  -0.5907
  -0.4302
  -0.2267

w1 =

  -0.2209
  -0.4194
  -0.5761
  -0.6761
  -0.7105
  -0.6761
  -0.5761
  -0.4194
  -0.2209

w1 =

  -0.2153
  -0.4090
  -0.5620
  -0.6597
  -0.6933
  -0.6597
  -0.5620
  -0.4090
  -0.2153

w1 =

  -0.2099
  -0.3988
  -0.5482
  -0.6437
  -0.6765
  -0.6437
  -0.5482
  -0.3988
  -0.2099

w1 =

  -0.2046
  -0.3889
  -0.5347
  -0.6280
  -0.6601
  -0.6280
  -0.5347
  -0.3889
  -0.2046

w1 =

  -0.1996
  -0.3793
  -0.5216
  -0.6127
  -0.6441
  -0.6127
  -0.5216
  -0.3793
  -0.1996

w1 =

  -0.1946
  -0.3699
  -0.5088
  -0.5978
  -0.6284
  -0.5978
  -0.5088
  -0.3699
  -0.1946

w1 =

  -0.1898
  -0.3608
  -0.4963
  -0.5832
  -0.6131
  -0.5832
  -0.4963
  -0.3608
  -0.1898

w1 =

  -0.1851
  -0.3519
  -0.4842
  -0.5689
  -0.5981
  -0.5689
  -0.4842
  -0.3519
  -0.1851

w1 =

  -0.1805
  -0.3433
  -0.4723
  -0.5550
  -0.5835
  -0.5550
  -0.4723
  -0.3433
  -0.1805

w1 =

  -0.1761
  -0.3348
  -0.4607
  -0.5415
  -0.5693
  -0.5415
  -0.4607
  -0.3348
  -0.1761

w1 =

  -0.1718
  -0.3266
  -0.4494
  -0.5282
  -0.5554
  -0.5282
  -0.4494
  -0.3266
  -0.1718

w1 =

  -0.1675
  -0.3186
  -0.4384
  -0.5153
  -0.5418
  -0.5153
  -0.4384
  -0.3186
  -0.1675

w1 =

  -0.1634
  -0.3108
  -0.4277
  -0.5027
  -0.5286
  -0.5027
  -0.4277
  -0.3108
  -0.1634

w1 =

  -0.1594
  -0.3032
  -0.4172
  -0.4904
  -0.5156
  -0.4904
  -0.4172
  -0.3032
  -0.1594

w1 =

  -0.1555
  -0.2957
  -0.4070
  -0.4784
  -0.5030
  -0.4784
  -0.4070
  -0.2957
  -0.1555

w1 =

  -0.1517
  -0.2885
  -0.3971
  -0.4667
  -0.4907
  -0.4667
  -0.3971
  -0.2885
  -0.1517

w1 =

  -0.1480
  -0.2814
  -0.3873
  -0.4553
  -0.4787
  -0.4553
  -0.3873
  -0.2814
  -0.1480

w1 =

  -0.1443
  -0.2745
  -0.3779
  -0.4442
  -0.4670
  -0.4442
  -0.3779
  -0.2745
  -0.1443

w1 =

  -0.1408
  -0.2678
  -0.3686
  -0.4333
  -0.4556
  -0.4333
  -0.3686
  -0.2678
  -0.1408

w1 =

  -0.1374
  -0.2613
  -0.3596
  -0.4227
  -0.4445
  -0.4227
  -0.3596
  -0.2613
  -0.1374

w1 =

  -0.1340
  -0.2549
  -0.3508
  -0.4124
  -0.4336
  -0.4124
  -0.3508
  -0.2549
  -0.1340

w1 =

  -0.1307
  -0.2486
  -0.3422
  -0.4023
  -0.4230
  -0.4023
  -0.3422
  -0.2486
  -0.1307

w1 =

  -0.1275
  -0.2425
  -0.3338
  -0.3924
  -0.4126
  -0.3924
  -0.3338
  -0.2425
  -0.1275

w1 =

  -0.1244
  -0.2366
  -0.3257
  -0.3828
  -0.4025
  -0.3828
  -0.3257
  -0.2366
  -0.1244

w1 =

  -0.1213
  -0.2308
  -0.3177
  -0.3735
  -0.3927
  -0.3735
  -0.3177
  -0.2308
  -0.1213

w1 =

  -0.1184
  -0.2252
  -0.3099
  -0.3643
  -0.3831
  -0.3643
  -0.3099
  -0.2252
  -0.1184
\end{matlaboutput}
\end{multicols*}
\begin{figure}[ht]
    \centering
    \includegraphics[width=0.8\textwidth]{im6.pdf}
    \caption{Solución de la EDP}
    \label{fig:solucion_edp6}
\end{figure}
