Es un ejemplo de uso del método de Adams-Bashforth de orden 3 para resolver la ecuación diferencial ordinaria (EDO):
\begin{equation*}
    y'=t-ty \quad y(0)=\frac{1}{2}\quad t\in[0,1]
\end{equation*}
Este método utiliza el resultado de un método de Runge-Kutta de orden 4 para calcular el $w_1$ y luego aplica el método de Adams-Bashforth para los siguientes pasos. Por ese motivo, se utiliza la función \texttt{runge11.m} y \texttt{ff.m} que se definió en el informe 8 y 10 respectivamente.

Sea crea un nuevo programa con el nombre de \textbf{bashforthtres11.m}

\textbf{Código en Matlab/Octave:}
\begin{matlabcode}
    %ejecuta adams-bashforth de orden 3
    y0=1/2; %w1=? por R-K
    a=0;b=1;n=4;k=2;%k=2 porque es de orden 3
    %---------------------
    h=(b-a)/n;
    yx=runge11(a,b,h,k,y0);
    t=a+k*h;
    for i=k+1:n
        Wx=yx(i)+h*(23*ff(t,yx(i))-16*ff(t-h,yx(i-1))+5*ff(t-2*h,yx(i-2)))/12;
        yx(i+1)=Wx;
        t=t+h;
    end
    yx
\end{matlabcode}
\textbf{Manual:}
\begin{itemize}
    \item Abrir el archivo \textbf{bashforthtres11.m} en Matlab/Octave.
    \item En el programa \texttt{ff.m} se define la EDO a resolver.
    \item En las lineas 
    \begin{verbatim}
y0=1/2; %w1=? por R-K
a=0;b=1;n=4;k=2;%k=2 porque es de orden 3
    \end{verbatim}
    se definen los parámetros de la EDO a resolver
    \item En la línea \textbf{n=4;} se define el número de pasos a realizar.
    \item Ejecutar el script con el comando \textbf{bashforthtres11} en la consola de Matlab/Octave.
\end{itemize}
\textbf{Corrida del Programa:}
\begin{matlaboutput}
    >> bashforthtres11

    yx =
    
        0.5000    0.5154    0.5588    0.6241    0.6983
\end{matlaboutput}
