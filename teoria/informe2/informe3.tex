Este programa resuelve un problema de contorno para una ecuación diferencial lineal de segundo orden:
\[
y'' + 3y' + 2y = t
\]
usando el método del disparo lineal. Este método convierte un problema de contorno en un conjunto de problemas de valor inicial que se resuelven con técnicas numéricas, como Runge-Kutta. El resultado se compara con la solución exacta conocida del problema.


\textbf{Variables:}

\begin{itemize}
    \item \texttt{a}, \texttt{b}: Extremos del intervalo \([a,b]\), donde se resuelve la ecuación diferencial.
    \item \texttt{n}: Número de subintervalos para el método numérico.
    \item \texttt{h}: Tamaño del paso, calculado como \( h = \frac{b-a}{n} \).
    \item \texttt{t}: Variable que recorre el intervalo \([a,b]\), utilizada tanto en la integración como en la solución exacta.
    \item \texttt{x0}, \texttt{y0}: Condiciones iniciales para \(x\) e \(y\) en el método del disparo.
    \item \texttt{k}: Parámetro que se pasa a la función diferencial definida en \texttt{ff.m}, dependiendo de la proporción que se está resolviendo.
    \item \texttt{yy}, \texttt{y11}, \texttt{y22}: Vectores con las soluciones numéricas obtenidas de las dos proporciones resueltas mediante el método de Runge-Kutta.
    \item \texttt{ys}: Solución lineal combinada del método de disparo, ajustada para satisfacer la condición de frontera.
    \item \texttt{yv}: Solución exacta del problema, usada para comparar contra \texttt{ys}.
    \item \texttt{c1}, \texttt{c2}: Constantes calculadas para definir la solución exacta del problema.
\end{itemize}

\textbf{Programa en Matlab/Octave:}
\begin{matlabcode}
%disparo lineal y''+3y'+2y=t
a=0;b=1;n=10;h=(b-a)/n;t=a;
%y1(x) es la primera proporción
k=1;x0=0;y0=1;
yy=runge_sis(a,b,x0,0,n,1);
y11=yy(1,:);
%y2(x) la segunda proporción
k=0;yy=runge_sis(a,b,0,01,n,0);
y22=yy(1,:);
ys=y11+(y0-y11(n+1))*y22/y22(n+1);
ys'
t=a:h:b;
c2=(5-3*exp(-2))/(4*(exp(-1)-exp(-2)));c1=3/4-c2;
yv=c1*exp(-2*t)+c2*exp(-t)+t/2-3/4;%solucion exacta
plot(t,ys,t,yv)
% 
% 
%
\end{matlabcode}

\textbf{Manual:}

1. Este archivo debe guardarse como \texttt{disparo\_lineal.m}.

2. Requiere tener definido el archivo \texttt{ff.m}, que contenga la ecuación diferencial como función de entrada.

3. También requiere el archivo \texttt{runge\_sis.m}, que implementa el método de Runge-Kutta para sistemas.

4. Se ejecuta desde la consola con:
\begin{verbatim}
>> disparo_lineal
\end{verbatim}

5. El programa grafica la solución numérica aproximada y la solución exacta, y además imprime la solución aproximada \texttt{ys} en consola.

\textbf{Corrida del Programa:}
\begin{matlaboutput}
octave:4> disparo_lineal
ans =

        0
   0.2982
   0.4959
   0.6303
   0.7251
   0.7950
   0.8492
   0.8938
   0.9324
   0.9673
   1.0000

\end{matlaboutput}
\begin{figure}[ht]
\centering
\includegraphics[width=0.8\textwidth]{im1_2.pdf}
\caption{Gráfico de la solución del disparo lineal.}
\end{figure}
