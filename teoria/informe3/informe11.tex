El método Crank-Nicolson es un esquema numérico implícito de segundo orden para resolver ecuaciones diferenciales parciales, especialmente la ecuación de difusión o calor. Combina los métodos de Euler hacia adelante y hacia atrás, proporcionando mayor estabilidad y precisión.

\textbf{Variables:}

\begin{itemize}
    \item $a$: coeficiente de difusión (en este caso $a = -1$).
    \item $h$: tamaño de paso espacial.
    \item $k$: tamaño de paso temporal.
    \item $x$: vector de posiciones espaciales.
    \item $w0$: condición inicial de la solución.
    \item $BB$: matriz que almacena las soluciones en cada paso temporal.
    \item $L$: parámetro auxiliar para el método.
    \item $A$, $B$: matrices del sistema lineal generado por el método.
    \item $wi$: vector auxiliar para la iteración.
    \item $cc$: solución del sistema en cada paso temporal.
\end{itemize}

\textbf{Programa en Matlab/Octave:}
\begin{matlabcode}
%07/07/2025
% Crank-Nicolson para un problema en especifico
%u_t=1*u_xx, u(t,0)=u(t,1) , u(0,x)=x-x^2
a=-1;h=1/4;k=(1/5);
format rat
BB=[];
x=0:1/4:1;
w0=x-x.^2;
w0=w0(2:4)';
BB(:,1)=[0 w0' 0]';
L=a^2*k/h^2;
A=[1+L -L/2 0;-L/2 1+L -L/2;0 -L/2 1+L]
B=[1-L L/2 0;L/2 1-L L/2;0 L/2 1-L]
%Mecanica de solucion
wi=B*w0;
format
cc=A\wi;w0=cc;BB(:,2)=[0 w0' 0]';
wi=B*w0;
format
cc=A\wi;w0=cc;BB(:,3)=[0 w0' 0]';
wi=B*w0;
format
cc=A\wi;w0=cc;BB(:,4)=[0 w0' 0]';
wi=B*w0;
format
cc=A\wi;w0=cc;BB(:,5)=[0 w0' 0]';
wi=B*w0;
format
cc=A\wi;w0=cc;BB(:,6)=[0 w0' 0]';
wi=B*w0;
format
cc=A\wi;w0=cc;BB(:,7)=[0 w0' 0]';
mesh(BB)
\end{matlabcode}

\textbf{Manual:}

Para ejecutar el programa, siga estos pasos:

1. Abra Matlab u Octave y copie el código proporcionado en un archivo llamado `crank\_nicolson.m`.
2. Ejecute el archivo en la consola con el comando `crank\_nicolson`.
3. El programa calculará la solución numérica de la ecuación de difusión usando el método Crank-Nicolson y mostrará la evolución de la solución en una gráfica de malla (`mesh`).
4. Los resultados intermedios de las matrices $A$ y $B$ se muestran en la salida, y la figura generada (`im11.pdf`) representa la solución aproximada en los distintos pasos temporales.

Asegúrese de tener instalado Matlab u Octave y de que el archivo se encuentre en el directorio de trabajo.
\textbf{Corrida del Programa:}
\begin{matlaboutput}
A =

       21/5       -8/5          0
       -8/5       21/5       -8/5
          0       -8/5       21/5

B =

      -11/5        8/5          0
        8/5      -11/5        8/5
          0        8/5      -11/5
\end{matlaboutput}
\begin{figure}
    \centering
    \includegraphics[width=0.8\textwidth]{im11.pdf}
    \caption{Solución de la EDP}
    \label{fig:solucion_edp11}
\end{figure}
