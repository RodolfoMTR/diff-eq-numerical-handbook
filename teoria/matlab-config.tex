% matlab-config.tex
\usepackage{listings}
\usepackage{xcolor}

% Estilo para código MATLAB
\lstdefinestyle{matlabstyle}{
    language=Matlab,
    basicstyle=\ttfamily\small,
    keywordstyle=\color{blue}\bfseries,
    commentstyle=\color{green!50!black}\itshape,
    stringstyle=\color{purple},
    numbers=left,
    numberstyle=\tiny\color{gray},
    stepnumber=1,
    numbersep=8pt,
    backgroundcolor=\color{gray!7},
    frame=single,
    frameround=tttt,
    rulecolor=\color{gray!20},
    breaklines=true,
    breakatwhitespace=false,
    showstringspaces=false,
    tabsize=4,
    captionpos=b,
    xleftmargin=1em,
    xrightmargin=1em
}

% Estilo para salida de consola/output
\lstdefinestyle{outputstyle}{
    basicstyle=\ttfamily\small\color{black!70},
    numbers=none,
    backgroundcolor=\color{white},
    frame=single,
    frameround=tttt,
    rulecolor=\color{gray!30},
    breaklines=true,
    breakatwhitespace=false,
    showstringspaces=false,
    tabsize=4,
    xleftmargin=1em,
    xrightmargin=1em
}

% Para archivo externo
\newcommand{\matlabfile}[2][]{%
  \lstinputlisting[style=matlabstyle,caption={#1}]{#2}%
}

% Definir entornos
\lstnewenvironment{matlabcode}
  {\lstset{style=matlabstyle}}
  {}

\lstnewenvironment{matlaboutput}
  {\lstset{style=outputstyle}}
  {}
