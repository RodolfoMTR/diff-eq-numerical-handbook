Este informe presenta la resolución numérica de una ecuación diferencial parcial (EDP) utilizando el método de diferencias finitas. El objetivo es aproximar la solución de la EDP mediante la discretización del dominio y la aplicación de un algoritmo iterativo implementado en Matlab/Octave.

\textbf{Variables:} \\
\begin{itemize}
    \item \texttt{la}: Parámetro relacionado con el coeficiente de difusión y el tamaño de la malla.
    \item \texttt{A}: Matriz de coeficientes del sistema lineal resultante de la discretización.
    \item \texttt{n}: Número de intervalos en la discretización espacial.
    \item \texttt{h}: Tamaño de paso espacial.
    \item \texttt{x}: Vector de puntos espaciales.
    \item \texttt{w0}: Vector de condiciones iniciales.
    \item \texttt{B}: Matriz que almacena la evolución de la solución en el tiempo.
    \item \texttt{w1}: Vector auxiliar para la actualización de la solución en cada paso temporal.
\end{itemize}

\textbf{Programa en Matlab/Octave:}
\begin{matlabcode}
clear all;
la=(4*1/1600)/(0.1^2);A=[];n=10;h=(2-0)/n;
A(1,1)=1-2*la;A(1,2)=la;
for i=2:n-2
    A(i,i-1)=la;A(i,i)=1-2*la;A(i,i+1)=la;
end
A(n-1,n-2)=la;A(n-1,n-1)=1-2*la;
x=0:h:2;
w0=-x.^2+2*x;w0=w0';B=[];B(:,1)=w0;
w0=w0(2:n);
for i=1:10*10
    w1=A*w0;
    B(:,i+1)=[0 w1' 0]';
    w0=w1;
end
mesh(B)
\end{matlabcode}

\textbf{Manual:}

Para ejecutar el programa, siga estos pasos:

1. Abra Matlab o GNU Octave.
2. Copie el código proporcionado en un archivo llamado `dif\_finitas.m`.
3. Ejecute el archivo con el comando:
    ```
    dif\_finitas
    ```
4. El programa generará una gráfica 3D (`mesh`) que muestra la evolución de la solución de la EDP en el tiempo.

Asegúrese de tener los permisos necesarios para guardar archivos y visualizar gráficos en su entorno de trabajo.
\textbf{Corrida del Programa:}
\begin{figure}[ht]
    \centering
    \includegraphics[width=0.8\textwidth]{im8.pdf}
    \caption{Solución de la EDP}
    \label{fig:solucion_edp8}
\end{figure}
