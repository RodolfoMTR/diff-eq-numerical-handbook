El objetivo de este informe es mostrar cómo graficar la solución de una ecuación diferencial ordinaria (EDO) utilizando Matlab/Octave. Para ello, se resolverá una EDO específica y se graficará su solución junto con algunos puntos evaluados en la misma.

\textbf{Problema a Resolver:}
\begin{equation}
y'+2y=x\quad ,\, y(0)=0\quad x\in[0,1]
\end{equation}
La solución de la ecuación diferencial es
\begin{equation}
y=\frac{2x-1}{4}+\frac{e^{-2x}}{4}
\end{equation}
Queremos encontrar los valores en $y(1),\,y(1/2)\,,y(\pi/4)$, de ahí, graficar junto a la solución en \textbf{Matlab/octave}. Para ese creamos el siguiente programa:
\begin{matlabcode}
y = @(x) (2*x-1)/4+exp(-2*x)/4; %definimos la función solución de la EDO
% evaluamos la función en los puntos deseados
y(1)
y(1/2)
y(pi/4)
\end{matlabcode}
Guardamos el programa con el nombre de su preferencia y lo ejecutamos, nos da el siguiente resultado en consola.
\begin{matlaboutput}
    ans = 0.2838
    ans = 0.091970
    ans = 0.1947    
\end{matlaboutput}
Graficamos la solución en el intervalo $[0,1]$ con los puntos calculados, para eso abajo del codigo del último programa escribimos estas dos lineas de codigo, las cuales al ejecutar el programa nos devolvera la siguiente gráfica.
\begin{matlabcode}
x=0:0.01:1;
plot(x,y(x),1,y(1),'ro',1/2,y(1/2),'ro',pi/4,y(pi/4),'ro')
\end{matlabcode}
\begin{figure}[ht]
\centering
\includegraphics[width=7cm]{figure_0.pdf}
\caption{Gráfica de la solución}
\end{figure}
